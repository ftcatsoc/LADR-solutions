\section{8.A Generalized Eigenvectors and Nilpotent Operators}

\begin{theorem}[5.10/8.13]\label{thm 5.10/8.13} 特征向量/广义特征向量的独立性 \:
    设\(V\)是有限维向量空间,\(T \in L(V)\).

    \(\lambda_1,\cdots,\lambda_m\)是\(T\)的特征值,\(v_1,\cdots,v_m\)是分别与之对应的(广义)特征向量.
    
    求证:\(v_1,\cdots,v_m\)线性无关.
\end{theorem}

\begin{proof}[证明5.10]
    令\(\sum_{i=1}^m a_iv_i=0\),下证\(a_1=\cdots=a_m=0\).

    设\(k \in \{1,\cdots,m\}\).对等式两边施加算子\(\prod_{i=1}^m (T-\lambda_i I)(i \ne k)\).
        \begin{align*}
            0=a_{k}\prod_{i=1}^m (T-\lambda_i I) v_{k}=
            a_{k}\prod_{i=1}^m (\lambda_{k}-\lambda_i) v_{k}
        \end{align*}
    由于这些特征值各不相同,故\(\forall i=1,\cdots,m,\lambda_i-\lambda_{k} \ne 0\),从而只能有\(a_{k}=0\).

    让\(k\)依次等于\(1,\cdots,m\),则\(a_1=\cdots=a_m=0\),证毕.
\end{proof}

\begin{proof}[证明8.13]
    令\(\sum_{i=1}^m a_iv_i=0\),下证\(a_1=\cdots=a_m=0\).

    设\(k \in \{1,\cdots,m\}\).
    令\(k\)为最大的使\((T-\lambda_{k})^k v_{k} \ne 0\)的自然数.
    
    令\(v_0=(T-\lambda_{k})^k v_{k}\),则\((T-\lambda_{k})^{k+1} v_{k}=0\).
    从而\(Tv_0=\lambda_{k}v_0\),进一步有
        \begin{align*}
            (T-\lambda I)v_0=(\lambda_{k}-\lambda_i)v_0 \Rightarrow 
            (T-\lambda I)^{\dim V} v_0=(\lambda_{k}-\lambda_i)^{\dim V}v_0
        \end{align*}
    对等式两边施加算子\(\prod_{i=1}^m (T-\lambda_i I)^{\dim V}(T-\lambda_{k})^k(i \ne k)\).
        \begin{align*}
            0=a_{k}\prod_{i=1}^m (T-\lambda_i I)^{\dim V}(T-\lambda_{k})^k v_{k}
            =a_{k}\prod_{i=1}^m (\lambda_{k}-\lambda_i)^{\dim V} v_0
        \end{align*}
    由于这些特征值各不相同,故\(\forall i=1,\cdots,m,(\lambda_i-\lambda_{k})^{\dim V} \ne 0\).
    从而只能有\(a_{k}=0\).
\end{proof}

{\kaishu 两者的证明思路大体相似,但定理8.13需要多乘以一个算子来构造一个“狭义”特征向量,以使得定理5.10所施加的算子可操作.}

\begin{theorem}[8.19]\label{thm 8.19} 幂零算子的矩阵 \:
    设\(V\)是有限维向量空间且\(N \in L(V)\)是幂零算子.

    求证:存在\(V\)的一组基,使得\(M(N)\)是对角线元素均为\(0\)的上三角矩阵.
\end{theorem}

\begin{proof}
    设\(\dim V=n\),并令\(B_1=\{v_1^1,\cdots,v_{n_1}^1\}\)是\(\ker N\)的一组基.

    由于\(\forall i=1,\cdots,n,\ker N^i \subseteq \ker N^{i+1}\),故依次扩充该基为\(\ker N^2,\cdots,\ker N^n\)的一组基.
    
    从\(\ker N^{i-1}\)扩充至\(\ker N^i\)时,所添加的向量组为\(B_i=\{v_{n_{i-1}+1}^i,\cdots,v_{n_i}^i\},i=1,\cdots,n\).
    
    因\(\ker N^i=\altspan(B_1,\cdots,B_i)\)且\(\ker N^n=V\),故\(B_1,\cdots,B_n\)中向量顺序排列即得\(V\)的基.
    
    由于\(N(\ker N^{i+1}) \subseteq \ker N^i\),因此\(N \altspan(B_i) \subseteq \altspan(B_1,\cdots,B_{i-1})\).
    
    因此\(M(N,(B_1,\cdots,B_n))\)中,\(B_i\)列中的\(B_i,\cdots,B_n\)行元素均为\(0\).
    
    因此在基\(B_1,\cdots,B_n\)下,\(\forall v \in \altspan(B_i),Nv\)都不会由对角线及其之下的分量构成.
    
    以矩阵语言表述,也就是\(M(N)\)是对角线元素均为\(0\)的上三角矩阵.
\end{proof}

\newpage

\begin{problem}[2]\label{8.A.2}
    定义\(T \in L(C^2)\)为\(T(w,z)=(-z,w)\).给出\(T\)的所有广义特征空间.
\end{problem}

\begin{proof}
    先给出\(T\)的所有特征值.令\(T(w,z)=\lambda(w,z)=(-z,w)\),得到\(\lambda_1=i,\lambda_2=-i\).

    因此\((T-iI)^2(w,z)=-2(w-iz,z+iw),(T+iI)^2(w,z)=(w+iz,z-iw)\).
    
    最后\(G(i,T)=\ker (T-iI)^2=(iz,z),G(-i,T)=\ker (T+iI)^2=(-iz,z)\).
\end{proof}

\begin{problem}[3]\label{8.A.3}
    设\(T \in L(V)\)是可逆变换且\(\dim V=n\).证明:\(\forall \lambda \ne 0,G(\lambda,T)=G(\lambda^{-1},T^{-1})\).
\end{problem}

\begin{proof}
    使用数学归纳法.\(n=1\)时,根据习题5.C.9,\(E(\lambda,T)=E(\lambda^{-1},T^{-1})\),命题成立.%此处使用了硬编码
    
    再假设\(n-1\)的情况成立,考虑\(n\)的情况.令\(u=(T-\lambda)v=(T^{-1}-\lambda^{-1} I)v\).
    \begin{align*}
        &(T-\lambda)^n v=(T-\lambda I)^{n-1}(T-\lambda I)v=(T-\lambda I)^{n-1}u=0 \\
        &(T^{-1}-\lambda^{-1} I)^n v=(T^{-1}-\lambda^{-1} I)^{n-1}(T^{-1}-\lambda^{-1} I)v
            =(T^{-1}-\lambda^{-1} I)^{n-1}u=0
    \end{align*}
    根据归纳假设,两式等价,即\(G(\lambda,T)=G(\lambda^{-1},T^{-1})\).
\end{proof}

\begin{problem}[5]\label{8.A.5}
    设\(T \in L(V),v \in V,m \in N^*\),且满足\(T^{m-1}v \ne 0,T^m v=0\).
    
    求证:\(v,Tv,\cdots,T^{m-1}v\)线性无关.
\end{problem}

\begin{proof}
    令\(\sum_{i=0}^{m-1} a_iT^i v=0\).对等式两边施加算子\(T^{m-1}\).

    得到\(\sum_{i=0}^{m-1} a_iT^{i+m-1} v=a_0T^{m-1} v=0\),因此\(a_0=0\).
    
    接着依次对该等式两边施加算子\(T^{m-2},\cdots,T\),得到\(a_0=a_1=\cdots=a_{m-2}=0\).
    
    而\(T^{m-1}v \ne 0\),因此必有\(a_{m-1}=0\),从而\(a_0=a_1=\cdots=a_{m-1}=0\),证毕.
\end{proof}

\begin{problem}[9]\label{8.A.9}
    设\(S,T \in L(V)\)且\(ST\)是幂零算子,证明\(TS\)也是幂零算子.
\end{problem}

\begin{proof}
    \(\ker (TS)^{\dim V}=\ker (TS)^{\dim V+1}=\ker T(ST)^{\dim V}S=V\).

    其中\((ST)^{\dim V}=0\),因此第三个等号成立.
\end{proof}

\begin{problem}[11]\label{8.A.11}
    证明或给出反例:若\(T \in L(V)\)且\(\dim V=n\),则\(T^n\)可对角化.
\end{problem}

\begin{proof}
    反例:设\(V=C^2,T(z_1,z_2)=(z_1,0)\),则\(T^2(z_1,z_2)=(z_1,0)\).

    因此\(T^2\)只有一个特征值,无法对角化.
\end{proof}

\newpage

\begin{problem}[12]\label{8.A.12}
    设\(V\)是有限维复向量空间且\(N \in L(V)\).存在\(V\)的一组基\(v_1,\cdots,v_n\),

    使得\(M(N,(v_1,\cdots,v_n))\)是对角线元素为\(0\)的上三角矩阵.求证:\(N\)是幂零算子.
\end{problem}

\begin{proof}
    由于该矩阵是三角矩阵,则根据定理5.26有\(Nv_i \in \altspan (v_1,\cdots,v_i)\).

    进一步,由于对角线元素均为\(0\),因此\(Nv_i \in \altspan (v_1,\cdots,v_{i-1})\).
    
    因此\(\forall i=1,\cdots,m,N^m v_i=0\),即\(N\)是幂零算子,证毕.
\end{proof}

\begin{problem}[15]\label{8.A.15}
    设\(N \in L(V)\)满足\(\ker N^{\dim V-1} \ne \ker N^{\dim V}\).

    求证:\(N\)是幂零算子且\(\forall i=1,\cdots,\dim V,\dim \ker N^i=i\).
\end{problem}

\begin{proof}
    由于\(\ker N^{\dim V-1} \ne \ker N^{\dim V}\),故\(\dim \ker N^i\)随\(i\)在\([1,\dim V]\)上严格单调增长.

    因此\(\dim \ker N<\cdots<\dim \ker N^{\dim V} \leq \dim V\).
    
    得到\(\dim \ker N^{\dim V}=\dim V\),即\(N\)是幂零算子,且\(\dim \ker N^i=i\).    
\end{proof}

\begin{problem}[16]\label{8.A.16}
    设\(T \in L(V)\).求证:\(V=\Img T^0 \supset \Img T^1 \supset \cdots\)
\end{problem}

\begin{proof}
    \(\forall v \in V,T^{i+1}v \in \Img T^{i+1},T^{i+1}v=T^i(Tv) \in \Img T^i\).

    因此\(\forall i \in N^*,\Img T^{i+1} \subseteq \Img T^i\).特别地,\(T^0=I,\Img I=V\).
\end{proof}

\begin{problem}[17]\label{8.A.17}
    设\(T \in L(V),m \in N^*\)满足\(\Img T^m=\Img T^{m+1}\).
    
    求证:\(\forall i \in N^*,\Img T^{m+i}=\Img T^m\).
\end{problem}

\begin{proof}
    根据\probref{8.A.16},\(\forall i \in N^*,\Img T^{m+i} \subseteq \Img T^m\).
    下证\(\forall i \in N^*,\dim \Img T^{m+i}=\dim \Img T^m\).
    
    根据定理3.22,\(\forall i \in N^*,\dim \Img T^i=\dim V-\dim \ker T^i\),
    
    故\(\dim \Img T^m=\dim \Img T^{m+1}\)表明\(\dim \ker T^m=\dim \ker T^{m+1}\),即\(\ker T^m =\ker T^{m+1}\).
    
    得\(\forall i \in N^*,\dim \Img T^{m+i}=\dim V-\dim \ker T^{m+i}=\dim V-\dim \ker T^m=\dim \Img T^m\).
    
    从而根据习题2.C.1,\(\forall i \in N^*,\Img T^{i+1}=\Img T^i\).
    %此处使用了硬编码
\end{proof}

\begin{problem}[18]\label{8.A.18}
    设\(T \in L(V),\dim V=n\).求证:\(\Img T^n=\Img T^{n+1}=\cdots\)
\end{problem}

\begin{proof}
    根据定理3.22和定理8.4,\(\forall i>n \in N^*,\dim \Img T^{i+1}=\dim \Img T^i\).

    根据习题2.C.1和\probref{8.A.16},\(\forall i \in N^*,\Img T^{i+1}=\Img T^i\).
    %此处使用了硬编码
\end{proof}
% End: source/chapter_8/8.A.tex

