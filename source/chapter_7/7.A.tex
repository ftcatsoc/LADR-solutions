\section{7.A Self-Adjoint and Normal Operators}

\begin{theorem}[7.10]\label{thm 7.10} 伴随算子的矩阵 \:
    \(e_1,\cdots,e_m,f_1,\cdots,f_n\)分别是\(V,W\)的规范正交基.

    设\(T \in L(V,W)\),则在这两组基下的\(M(T)\)是\(M(T^*)\)的共轭转置,即\(\ol{M(T)^T}=M(T^*)\).
\end{theorem}

\begin{proof}
    考虑\(M(T)_{i,j}\).由于\(Te_j=\sum_{i=1}^n \ip*{Te_j}{f_i}f_i\),从而\(M(T)_{i,j}=\ip*{Te_j}{f_i}\).

    再考虑\(M(T^*)_{j,i}\).由于\(T^*f_i=\sum_{j=1}^n \ip*{T^*f_i}{e_j}e_j\),从而\(M(T^*)_{j,i}=\ip*{T^*f_i}{e_j}\).
    \begin{align*}
        M(T^*)_{j,i}=\ip*{T^*f_i}{e_j}=\ip*{f_i}{Te_j}=\ol{\ip*{Te_j}{f_i}}=\ol{M(T)_{i,j}}
    \end{align*}
    于是\(\ol{M(T)^T}=M(T^*)\),证毕.
\end{proof}

\begin{problem}[2]\label{7.A.2}
    设\(V\)是有限维内积空间且\(T \in L(V)\).

    证明:\(\lambda\)是\(T\)的特征值和\(\ol{\lambda}\)是\(T^*\)的特征值等价.
\end{problem}

\begin{proof}
    利用\(\dim \ker(T^*)=\dim \ker(T)\),有
    \begin{align*}
        0<\dim \ker(T-\lambda I)=\dim \ker(T-\lambda I)^*=\dim \ker(T^*-\ol{\lambda}I)
    \end{align*}
因此\(E(\ol{\lambda},T^*) \ne \{0\}\),即\(\ol{\lambda}\)是\(T^*\)的一个特征值.
\end{proof}

\begin{problem}[3]\label{7.A.3}
    设\(V\)是有限维内积空间且\(T \in L(V)\),\(U\)是\(V\)的一个子空间.

    求证:\(U\)在\(T\)下不变和\(U^\bot\)和\(T^*\)下不变等价.
\end{problem}

\begin{proof}
    由\(\forall u \in U,u' \in U^\bot\),则\(\ip*{u}{u'}=0\),考虑\(\ip*{Tu}{u'}\)和\(\ip*{u}{T^*u'}\).

    由\(\ip*{Tu}{u'}=0 \Leftrightarrow \ip*{u}{T^*u'}=0\),得到\(Tu \in U\)和\(T^*u' \in U^\bot\)等价.
\end{proof}

\begin{problem}[4]\label{7.A.4}
    设\(T \in L(V,W)\).求证:
    a.\(T\)单射和\(T^*\)满射等价. \quad b.\(T\)满射和\(T^*\)单射等价.
\end{problem}

\begin{proof}[证明a]
    \(\ker T=\{0\} \Leftrightarrow (\ker T)^\bot=V \Leftrightarrow \operatorname{Im} T^*=V\).
\end{proof}

\begin{proof}[证明b]
    \(\operatorname{Im} T=V \Leftrightarrow (\operatorname{Im} T)^\bot=\{0\} \Leftrightarrow \operatorname{Im} T^*=\{0\}\).
\end{proof}

\begin{problem}[11]\label{7.A.11}
    设\(P \in L(V)\)满足\(P^2=P\).
    
    求证:\(T\)是自伴算子等价于存在\(V\)的一个子空间\(U\),使得\(P=P_U\).
\end{problem}

\begin{proof}
    必要性:考虑\(U\)和\(U^\bot\).\(\forall v,w \in V,v=v_1+v_2,w=w_1+w_2,v_1,w_1 \in U,v_2,w_2 \in U^\bot\).
    \begin{align*}
        \ip*{Pv}{w}=\ip*{v_1}{w_1+w_2}=\ip*{v_1}{w_1}+\ip*{v_1}{w_2}=\ip*{v_1}{w_1}+\ip*{v_2}{w_1}=\ip*{v}{Pw}
    \end{align*}
    充分性:根据\probref{5.B.4},\(V= \ker P \oplus \operatorname{Im} P\).考虑\(\forall v=v_1+v_2,v_1 \in \ker P,v_2 \in \operatorname{Im} P\).
    \begin{align*}
        \ip*{Pv}{v}=\ip*{v_2}{v_1+v_2}=\ip*{v_2}{v_1}+\norm*{v_2}^2
        =\ip*{v_1}{v_2}+\norm*{v_2}^2=\ip*{v_1+v_2}{v_2}=\ip*{v}{Pv}
    \end{align*}
    因此\(\ip*{v_1}{v_2}=0\),即\(\ker P\)和\(\operatorname{Im} P\)互为正交补,取\(U=\operatorname{Im} P\)即可.
\end{proof}

\newpage

\begin{problem}[14]\label{7.A.14}
    设\(T \in L(V)\)是正规算子.\(v,w \in V\)满足\(Tv=3v,Tw=4w,\norm*{v}=\norm*{w}=2\).

    求\(\norm*{T(v+w)}\).
\end{problem}

\begin{proof}
    根据定理7.22,\(\ip*{v}{w}=0\),从而\(\norm*{T(v+w)}^2=\norm*{3v+4w}^2=\norm*{3v}^2+\norm*{4w}^2=100\).
    
    根据范数的正定性,显然有\(\norm*{T(v+w)}=10\).
\end{proof}

\begin{problem}[15]\label{7.A.15}
    给定\(u,x \in V\),定义\(T \in L(V)\)为\(\forall v \in V,Tv=\ip*{v}{u}x\).

    a.当\(F=R\)时,证明:\(T\)是自伴算子等价于\(\exists \lambda \ne 0 \in R,x=\lambda u\).
    
    b.证明:\(T\)是正规算子等价于\(\exists \lambda \ne 0,x=\lambda u\).
\end{problem}

\begin{proof}[证明a]
    \(\ip*{v}{T^*w}=\ip*{Tv}{w}=\ip*{v}{u}\ip*{x}{w}=\ip*{v}{\ip*{x}{w}u},T^*w=\ip*{x}{w}u\).

    必要性:\(\ip*{Tv}{w}=\ip*{\lambda \ip*{v}{u}}{w}u=\lambda \ip*{v}{u} \ip*{u}{w}\)
    \(=\ip*{v}{\lambda \ip*{w}{u}u}=\ip*{v}{Tw}\).
    
    充分性:\(\ip*{x}{w}u=T^*w=Tw=\ip*{w}{u}x\).取\(w=u\),得\(\lambda=\dfrac{\ip*{x}{x}}{\ip*{x}{u}}\).
\end{proof}

\begin{proof}[证明b]
    分别给出\(T^*T\)和\(TT^*\).
    \begin{align*}
        &(T^*T)v=T^*(\ip*{v}{u}x)=\ip*{v}{u}T^*x=\norm*{x}^2\ip*{v}{u}u \\
        &(TT^*)v=T(\ip*{x}{v}u)=\ip*{x}{v}Tu=\norm*{u}^2\ip*{x}{v}x
    \end{align*}
    必要性:\((T^*T)v=\norm*{\lambda u}^2 \ip*{v}{u}u=\lambda \ol{\lambda} \norm*{u}^2 \ip*{v}{u}u\)
    \(=\norm*{u}^2 \ip*{v}{\lambda u}\lambda u=(TT^*)v\).

    充分性:\(\norm*{x}^2\ip*{v}{u}u=\norm*{u}^2\ip*{x}{v}x\).取\(v=x\),得\(\lambda=\dfrac{\ip*{x}{u}}{\ip*{u}{u}}\).
\end{proof}

\begin{problem}[16]\label{7.A.16}
    设\(T \in L(V)\)是正规算子.求证:\(\ker T^*=\ker T,\operatorname{Im} T^*=\operatorname{Im} T\).
\end{problem}

\begin{proof}
    \(\forall v \in V,Tv=0 \Leftrightarrow \norm*{Tv}=0 \Leftrightarrow \norm*{T^*v}=0 \Leftrightarrow T^*v=0\).

    \(\operatorname{Im} T^*=(\ker T)^\bot=(\ker T^*)^\bot=\operatorname{Im} T\).

    \(V=\ker T \oplus (\ker T)^\bot=\ker T \oplus \operatorname{Im} T^*=\ker T \oplus \operatorname{Im} T\).
\end{proof}

\begin{problem}[17]\label{7.A.17}
    设\(T \in L(V)\)是正规算子.求证:\(\forall n \in N^*,\ker T^n=\ker T,\operatorname{Im} T^n=\operatorname{Im} T\).
\end{problem}

\begin{proof}
    先证明\(\ker T^2 \subseteq \ker T\).考虑\(v \in \ker T^2\),则\(\ip*{T^2 v}{T^2 v}=0\).
    \begin{align*}
        &\ip*{T^2 v}{T^2 v}=\ip*{Tv}{T^*T^2v}=\ip*{Tv}{T^2T^*v}=\ip*{T^*Tv}{TT^*v}=0 \Rightarrow T^*Tv=0 \\
        &\ip*{T^*Tv}{v}=\ip*{Tv}{Tv}=0 \Rightarrow Tv=0 \Rightarrow v \in \ker T
    \end{align*}
    结合显然的\(\ker T \subseteq \ker T^2\)有\(\ker T^2=\ker T\).

    根据定理8.3和\probref{8.A.17},有\(\forall n \in N^*,\ker T^n=\ker T,\operatorname{Im} T^n=\operatorname{Im} T\).
\end{proof}
% End: source/chapter_7/7.A.tex

