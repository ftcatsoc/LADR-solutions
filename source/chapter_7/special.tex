\section{Special Exercises for Chapter 6-7}

\begin{problem}[6.A.29]\label{6.A.29}
    对于\(u,v \in V\),定义\(d(u,v)=\norm*{u-v}\).

    a.证明\(d\)是\(V\)上的测度.
    
    b.设\(V\)是有限维内积空间,证明\(d\)是\(V\)上的完备度量.
\end{problem}

\begin{proof}[证明a]
    正定性:\(d(u,v)=\norm*{u-v} \geq 0\),当且仅当\(u=v\)时\(\norm*{u-v}=0\),即\(d(u,v)=0\).

    对称性:\(d(u,v)=\norm*{u-v}=\norm*{(-1)(v-u)}=\abs*{-1}\norm*{v-u}=d(v,u)\).
    
    三角不等式:\(d(u,w)=\norm*{(u-v)+(v-w)} \leq \norm*{u-v}+\norm*{v-w}=d(u,v)+d(v,w)\).
\end{proof}

\begin{proof}[证明b]
    设\(e_1,\cdots,e_n\)是\(V\)的一组规范正交基.考虑柯西收敛的实数列\(\{c_k^i\},i=1,\cdots,n\).

    由实数的完备性,\(\lim_{k \rightarrow +\infty}c_k^i=c^i \in R\),于是考虑构造向量列\(\{c_k^ie_i\} \in V\).
    
    下证其在测度\(d\)下,\(\{c_k^ie_i\}\)一定收敛至\(c^ie_i \in V\).
    \begin{align*}
        \forall \varepsilon>0,\exists N_i \in N^*,\forall k>N_i,
        d(c_k^ie_i,c^ie_i)=\norm*{c_k^ie_i-c^ie_i}=\abs*{c_k^i-c^i}<\varepsilon
    \end{align*}
    下面考虑向量列\(\{v_k\},v_k=\sum_{i=1}^n \ip*{v_k}{e_i}e_i\),令\(\ip*{v_k}{e_i}=c_k^i\),定义\(v=\sum_{i=1}^n c^ie_i\).
    \begin{align*}
        \norm*{v_k-v}^2=\norm*{\sum_{i=1}^n(c_k^ie_i-c^ie_i)}^2=\norm*{\sum_{i=1}^n(c_k^i-c^i)e_i}^2
        =\sum_{i=1}^n(c_k^i-c^i)^2
    \end{align*}
    由于\(\forall i=1,\cdots,n,\forall \varepsilon>0,\exists N_i \in N^*,\forall k>N_i,\)
    \(\abs*{c_k^i-c^i}<\dfrac{\varepsilon}{\sqrt{n}}\).取\(N=\max\{N_1,\cdots,N_n\}\).
    
    于是\(\forall k>N,\norm*{v_k-v}^2=\sum_{i=1}^n(c_k^i-c^i)^2<n(\dfrac{\varepsilon}{\sqrt{n}})^2=\varepsilon^2\),
    即\(\norm*{v_k-v}<\varepsilon\).
\end{proof}

\begin{lemma}[6.B.16]\label{lem 6.B.16}
    设\(V\)是有限维复向量空间,满足\(\dim V=n,T \in L(V)\).

    最大行和范数\(\norm*{T}_{\infty}=\max_{1 \leq i \leq n} \sum_{j=1}^n \abs*{M(T)_{i,j}}\)和谱范数\(\norm*{T}_2\)满足
    \(\norm*{T}_2 \leq \sqrt{n}\norm*{T}_{\infty}\).
\end{lemma}

\begin{proof}
    设\(e_1,\cdots,e_n\)是\(V\)的一组规范正交基.
    下证\(\forall v=\sum_{i=1}^n \ip*{v}{e_i}e_i \in V,\norm*{Tv}^2 \leq n \norm*{Tv}_{\infty}^2\).
    \begin{align*}
        \norm*{Tv}^2=\abs*{\sum_{i=1}^n \ip*{v}{e_i}Te_i}^2 \leq \sum_{i=1}^n \ip*{v}{e_i}^2 \sum_{i=1}^n \norm*{Te_i}^2 
        =\norm*{v}^2 \sum_{i=1}^n \norm*{Te_i}^2
    \end{align*}
    不等号来自于柯西-施瓦兹不等式,考虑\(\sum_{i=1}^n \norm*{Te_i}^2\),设最大行和为第\(r\)行.
    \begin{align*}
        \norm*{Te_i}^2=\norm*{\sum_{j=1}^n M(T)_{i,j}e_j}^2=\sum_{j=1}^n M(T)_{i,j}^2
        \leq (\sum_{j=1}^n \abs*{M(T)_{i,j}})^2 \leq (\sum_{j=1}^n \abs*{M(T)_{r,j}})^2=\norm*{T}_{\infty}^2
    \end{align*}
    于是\(\sum_{i=1}^n \norm*{Te_i}^2 \leq n\norm*{Te_r}^2=n\norm*{T}_{\infty}^2\).将结果代入原式,得到
    \begin{align*}
        \norm*{T}_2^2=\sup_{v \ne 0} \dfrac{\norm*{Tv}^2}{\norm*{v}^2} 
        \leq \sup_{v \ne 0} \dfrac{n \norm*{v}^2 \norm*{T}_{\infty}^2}{\norm*{v}^2}=n \norm*{T}_{\infty}^2,
        \norm*{T}_2 \leq \sqrt{n}\norm*{T}_{\infty}
    \end{align*}
\end{proof}

\newpage

\begin{problem}[6.B.16]\label{6.B.16}
    设\(V\)是有限维复内积空间且\(T \in L(V)\),有特征值\(\lambda_1,\cdots,\lambda_n\).

    \(T\)的谱半径\(\rho(T)<1\).证明:\(\forall \varepsilon>0,\exists m \in N^*,\norm*{T^m}_2<\varepsilon\).    
\end{problem}

\begin{proof}
    \(V=\oplus_{i=1}^n G(\lambda_i,T)=\oplus_{i=1}^n\oplus_{j=1}^{n_i} J_j(\lambda_i,T)\),
    令\(\dim J_j(\lambda_i,T)=d_{i,j}\).

    考虑\(M(T^m|_{J_j(\lambda_i,T)})\),展开之,得到
    \begin{align*}
        M(T^m|_{J_j(\lambda_i,T)})=
        \begin{pmatrix}
            \lambda_i^m & C_m^1 \lambda_i^{m-1} & \cdots      & C_m^{m+1-d_{i,j}} \lambda_i^{m+1-d_{i,j}} \\
                        & \lambda_i^m           & \ddots      & \vdots                                    \\
                        &                       & \lambda_i^m & C_m^1 \lambda_i^{m-1}                     \\
                 0      &                       &             & \lambda_i^m
        \end{pmatrix}
    \end{align*}
    因此\(M(T^m|_{J_j(\lambda_i,T)})\)中的元素具有统一形式\(C_m^k \lambda_i^{m-k},k=0,\cdots,d_{i,j}-1\),考虑其绝对值.

    显然,由于\(\abs*{\lambda_i}<1\),故其最大值将在前半段取得.令\(a_k=C_m^k \abs*{\lambda_i}^{m-k},k=0,\cdots,d_{i,j}-1\).令
    \begin{align*}
        \dfrac{a_{k+1}}{a_k}=\dfrac{C_m^{k+1}\abs*{\lambda_i}^{m-(k+1)}}{C_m^k \abs*{\lambda_i}^{m-k}}=
        \dfrac{m-k}{k+1} \abs*{\lambda_i}^{-1}=1,k=\dfrac{m-\abs*{\lambda_i}}{1+\abs*{\lambda_i}}
    \end{align*}
    因此\(a_{\max}\)在\([\dfrac{m-\abs*{\lambda_i}}{1+\abs*{\lambda_i}}]\)或\(\{\dfrac{m-\abs*{\lambda_i}}{1+\abs*{\lambda_i}}\}\)处取得.
    随着\(m\)增长,将有\(\dfrac{m-\abs*{\lambda_i}}{1+\abs*{\lambda_i}}>d_{i,j}-1\).

    因此当\(m\)足够大时,数列在\([1,d_{i,j}-1]\)上单调增,其最大值将在最后一项取得,为
    \begin{align*}
        a_{d_{i,j}-1}=C_m^{d_{i,j}-1}\abs*{\lambda}^{m-(d_{i,j}-1)} \leq 
        \dfrac{m^{d_{i,j}-1}}{(d_{i,j}-1)!}\abs*{\lambda}^{m-(d_{i,j}-1)}
    \end{align*}
    选择最大行和范数\(\norm*{\cdot}_{\infty}\)作为算子范数,因而
    \begin{align*}
        \norm*{T^m|_{J_j(\lambda_i,T)}}_{\infty} \leq d_{i,j}\dfrac{m^{d_{i,j}-1}}{(d_{i,j}-1)!}\abs*{\lambda}^{m-(d_{i,j}-1)}
        =\dfrac{d_{i,j}}{(d_{i,j}-1)! \abs*{\lambda}^{d_{i,j}-1}}m^{d_{i,j}-1}\abs*{\lambda_i}^m
    \end{align*}
    由于\(\abs*{\lambda_i}<1\),令\(b_m=m^{d_{i,j}-1}\abs*{\lambda_i}^m\).
    下证\(\lim_{m \rightarrow +\infty} \dfrac{b_{m+1}}{b_m}<1\).
    \begin{align*}
        \lim_{m \rightarrow +\infty} \dfrac{b_{m+1}}{b_m}
        =\lim_{m \rightarrow +\infty} \abs*{\lambda_i}(1+\dfrac{1}{m})^{d_{i,j}-1}=\abs*{\lambda_i}<1,
        \lim_{m \rightarrow +\infty} b_m=0
    \end{align*}
    令\(\sum_{i=1}^n n_i=N\),则\(\forall \varepsilon>0,\exists m \in N^*,\)
    \(\norm*{T^m|_{J_j(\lambda_i,T)}}_{\infty}<\dfrac{\varepsilon}{N \sqrt{d_{\max}}}\),因此
    \begin{align*}
        \norm*{T^m}_{\infty} \leq \sum_{i=1}^n \sum_{j=1}^{n_i}\norm*{T^m|_{J_j(\lambda_i,T)}}<
        N \cdot \dfrac{\varepsilon}{N \sqrt{d_{\max}}}=\dfrac{\varepsilon}{\sqrt{d_{\max}}}
    \end{align*}
    由于\(\norm*{T^m|_{J_j(\lambda_i,T)}}_2 \leq \sqrt{d_{i,j}}\norm*{T^m|_{J_j(\lambda_i,T)}}_{\infty}\),
    故\(\norm*{T^m}_2 \leq \sqrt{d_{\max}}\norm*{T^m}_{\infty}<\varepsilon\).
\end{proof}
% End: source/chapter_7/special.tex

