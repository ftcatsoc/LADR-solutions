\section{7.B The Spectral Theorem}

\begin{theorem}[7.24]\label{thm 7.24} 复谱定理 \:
    设\(V\)是有限维复内积空间且\(T \in L(V),\dim V=n\).

    求证:\(T\)是正规算子等价于\(V\)有一组\(T\)的规范正交特征向量组成的基.
\end{theorem}

\begin{proof}
    必要性:若\(V\)有一组\(T\)的规范正交特征基,则\(M(T),M(T^*)\)都是对角矩阵.

    由于对角矩阵的可交换性,有\(M(TT^*)=M(T)M(T^*)=M(T^*)M(T)=M(T^*T)\).
    
    因此\(T^*T=TT^*\),\(T\)是正规算子.
    
    充分性:根据定理6.38,\(V\)有一组规范正交基\(e_1,\cdots,e_n\)使得\(M(T)\)为上三角矩阵.
    
    由于\(M(T^*)\)是\(M(T)\)的共轭转置,故\(M(T^*)\)是下三角矩阵.
    
    由于\(T\)是正规算子,故根据定理7.20,\(i=1,\cdots,n,\norm*{Te_i}^2=\norm*{T^*e_i}^2\).设\(M(T)_{i,j}=a_{i,j}\).
    \begin{align*}
        \norm*{Te_1}^2=\abs*{a_{1,1}}^2=\sum_{i=1}^n \abs*{a_{1,i}}^2=\norm*{T^*e_1}^2,a_{2,1}=\cdots=a_{n,1}=0
    \end{align*}
    进一步地,由\(\norm*{Te_2}^2=\norm*{T^*e_2}^2\)可以得到\(a_{3,2}=\cdots=a_{n,2}=0\).
    
    以此类推,\(M(T)\)的所有严格上三角部分元素和严格下三角部分元素均为\(0\),即\(T\)可对角化.
    
    此时的\(e_1,\cdots,e_n\)就是\(T\)的特征向量,它们构成了\(V\)的规范正交基.
\end{proof}

\begin{lemma}[7.28]\label{lem 7.28} 自伴算子和不变子空间 \:
    设\(T \in L(V)\)是自伴算子且\(U\)是\(T\)下的不变子空间,则

    a.\(U^\bot\)是\(T\)下的不变子空间. \quad
    b.\(T|_U\)是自伴算子. \quad
    c.\(T_{U^\bot}\)是自伴算子.
\end{lemma}

\begin{proof}[证明a]
    设\(\forall u \in U,u' \in U^\bot\),由于\(Tu \in U\),故\(\ip*{Tu}{u'}=\ip*{u}{Tu'}=0\),即\(Tu' \in U^\bot\).
\end{proof}

\begin{proof}[证明b]
    设\(\forall u_1,u_2 \in U\),则\(\ip*{T|_U u_1}{u_2}=\ip*{Tu_1}{u_2}=\ip*{u_1}{Tu_2}=\ip*{u_1}{T|_U u_2}\).
\end{proof}

\begin{proof}[证明c]
    设\(\forall u_1,u_2 \in U^\bot\),则\(\ip*{T|_{U^\bot}u_1}{u_2}=\ip*{Tu_1}{u_2}=\ip*{u_1}{Tu_2}=\ip*{u_1}{T|_{U^\bot}u_2}\).
\end{proof}

\begin{theorem}[7.29]\label{thm 7.29} 实谱定理 \:
    设\(V\)是有限维实内积空间且\(T \in L(V),\dim V=n\).

    求证:\(T\)是自伴算子等价于\(V\)有一组\(T\)的规范正交特征向量组成的基.
\end{theorem}

\begin{proof}
    必要性:若\(V\)有一组\(T\)的规范正交特征基,则\(M(T),M(T^*)\)都是对角矩阵.

    由对角矩阵转置不变,有\(M(T)=M(T)^T=M(T^*)\).因此\(T\)是自伴算子.
    
    充分性:使用数学归纳法,\(\dim V=1\)的情况显然成立.
    
    假设满足\(\dim U<\dim V\)的向量空间\(U\)上的自伴算子都有规范正交特征基.
    
    由于实内积空间上的自伴算子一定有特征值\footnote{这个结论将在复化章节中证明.},故设\(e_1\)满足\(\norm*{e_1}=1\)是\(T\)的特征向量.
    
    令\(U=\operatorname{span}(e_1)\),考虑\(U^\bot\).根据引理7.28,\(T|_{U^\bot}\)是自伴算子,
    
    故\(T|_{U^\bot}\)有规范正交特征基\(e_2,\cdots,e_n\),于是\(e_1,\cdots,e_n\)就是\(T\)的规范正交特征基.
\end{proof}

\newpage

\begin{problem}[4]\label{7.B.4}
    设\(V\)是有限维复内积空间且\(\dim V=n\),\(T \in L(V)\)有特征值\(\lambda_1,\cdots,\lambda_m\).

    求证:\(T\)是正规算子当且仅当\(V=\oplus_{i=1}^m E(\lambda_i,T)\)且特征空间两两正交.
\end{problem}

\begin{proof}
    必要性直接依据复谱定理得证.

    充分性:构造每个特征空间\(E(\lambda_i,T)\)的规范正交基\(v_1^i,\cdots,v_{n_i}^i\).
    
    将这些规范正交基合并,就得到了\(V\)的规范正交特征基,根据复谱定理,\(T\)是正规算子.
    
    实内积空间的版本的证明是类似的.
\end{proof}

\begin{problem}[6]\label{7.B.6}
    证明:复内积空间\(V\)上的正规算子\(T\)是自伴算子当且仅当其特征值都是实的.
\end{problem}

\begin{proof}
    根据复谱定理,\(V\)有一组规范正交特征基\(e_1,\cdots,e_n\).

    从而\(M(T)=\mathrm{diag}(\lambda_1,\cdots,\lambda_n),M(T^*)=\mathrm{diag}(\ol{\lambda_1},\cdots,\ol{\lambda_n})\).
    
    \(M(T)=M(T^*)\)等价于\(\forall i=1,\cdots,n,\lambda_i=\ol{\lambda_i}\),
    即\(\forall i=1,\cdots,n,\lambda_i\)都是实数,证毕.
\end{proof}

\begin{problem}[8]\label{7.B.8}
    设\(V\)是有限维复内积空间且\(T \in L(V)\)是正规算子,满足\(T^9=T^8\).

    求证:\(T\)是自伴算子且为幂等算子.
\end{problem}

\begin{proof}
    \(V\)应有一组规范正交特征基\(e_1,\cdots,e_n\),从而\(M(T)=\mathrm{diag}(\lambda_1,\cdots,\lambda_n)\).

    若\(T^9=T^8\),则\(\forall i=1,\cdots,n,\lambda_i^9=\lambda_i^8,\lambda_i^8(\lambda_i-1)=0\),
    即特征值只可能为\(0\)或\(1\).

    根据\probref{7.B.6},\(T\)是自伴算子;根据\probref{5.B.4},\(T\)是幂等算子.
\end{proof}

\begin{problem}[12]\label{7.B.12}
    设\(T \in L(V)\)为自伴算子,满足\(\forall \varepsilon>0,\exists \lambda \in F,\norm*{v}=1\)使得\(\norm*{Tv-\lambda v}<\varepsilon\).

    求证:存在\(T\)的一个特征值\(\mu\),满足\(\abs*{\lambda-\mu}<\varepsilon\).
\end{problem}

\begin{proof}
    由\(T\)是自伴算子,存在\(V\)的规范正交特征基\(e_1,\cdots,e_n\),使得\(M(T)=\mathrm{diag}(\lambda_1,\cdots,\lambda_n)\).

    因此\(\forall v \in V,\norm*{v}=1,v=\sum_{i=1}^n a_ie_i,Tv=\sum_{i=1}^n a_i\lambda_ie_i\),
    其中\(\sum_{i=1}^n \abs*{a_i}^2=1\).

    从而\(\norm*{Tv-\lambda v}^2=\norm*{\sum_{i=1}^n a_i(\lambda_i-\lambda)e_i}^2=\sum_{i=1}^n(\lambda_i-\lambda)^2 a_i^2<\varepsilon^2\).

    若\(\forall i=1,\cdots,n,\abs*{\lambda-\lambda_i}^2 \geq \varepsilon^2\),
    考虑\(\sum_{i=1}^n(\lambda_i-\lambda)^2 a_i^2 \geq \varepsilon^2 \sum_{i=1}^n a_i^2=\varepsilon^2\),矛盾.

    因此存在\(i=1,\cdots,n,\mu=\lambda_i\),使得\(\abs*{\mu-\lambda}<\varepsilon\).
\end{proof}

\newpage

\begin{problem}[20*]\label{7.B.20*} 正规算子和不变子空间 \:
    设\(T \in L(V)\)是正规算子且\(U\)是\(T\)下的不变子空间,

    a.\(U^\bot\)是\(T\)下的不变子空间. \quad
    b.\(T|_U \in L(U),T|_{U^\bot} \in L(U^\bot)\)都是正规算子.
\end{problem}

\begin{proof}[证明a]
    根据\probref{7.A.16},若\(T\)为正规算子,则\(\operatorname{Im} T=\operatorname{Im} T^*\).

    设\(\forall u \in U\),则\(T^*u \in \operatorname{Im} T^*=\operatorname{Im} T \subseteq U\),因此\(U\)也是\(T^*\)下的不变子空间.
    
    从而\(\forall u' \in U^\bot,\ip*{u}{Tu'}=\ip*{T^*u}{u'}=0\),因此\(Tu' \in U^\bot\),\(U^\bot\)是\(T\)下的不变子空间.
\end{proof}

\begin{proof}[证明b]
    \(\forall u \in U,\norm*{T|_U u}^2=\norm*{Tu}^2=\norm*{T^*u}^2=\norm*{T^*|_U u}^2\),即\(T|_U \in L(U)\)是正规算子;

    \(\forall u' \in U^\bot,\norm*{T|_{U^\bot}u'}^2=\norm*{Tu'}^2=\norm*{T^*u'}^2=\norm*{T^*|_{U^\bot}u'}^2\),
    即\(T|_{U^\bot} \in L(U^\bot)\)是正规算子.
\end{proof}

\begin{problem}[13]\label{7.B.13}
    沿用定理7.24的符号,但使用实谱定理的证明思路证明复谱定理.
\end{problem}

\begin{proof}
    必要性的证明与原来一致.

    充分性:使用数学归纳法,\(\dim V=1\)的情况显然成立.
    
    假设满足\(\dim U<\dim V\)的向量空间\(U\)上的自伴算子都有规范正交特征基.
    
    由于复向量空间上的算子一定有特征值,故设\(e_1\)满足\(\norm*{e_1}=1\)是\(T\)的特征向量.
    
    令\(U=\operatorname{span}(e_1)\),考虑\(U^\bot\).根据\probref{7.B.20*},\(T|_{U^\bot}\)是正规算子,
    
    故\(T|_{U^\bot}\)有规范正交特征基\(e_2,\cdots,e_n\),于是\(e_1,\cdots,e_n\)就是\(T\)的规范正交特征基.
\end{proof}

\begin{problem}[14]\label{7.B.14}
    设\(V\)是有限维实内积空间且\(T \in L(V)\).

    求证:\(V\)存在由\(T\)的特征向量组成的基等价于存在一种内积定义,使得\(T\)为自伴算子.
\end{problem}

\begin{proof}
    必要性:根据实谱定理,\(V\)有一组\(T\)的规范正交特征向量组成的基.

    充分性:将基设为\(e_1,\cdots,e_m\).考虑\(\forall i,j=1,\cdots,m,a_i,b_i \in R,\)
    \(v_1=\sum_{i=1}^m a_ie_i,v_2=\sum_{j=1}^m b_je_j\).
    
    由于\(V\)是实内积空间,故\(\ip*{v_1}{v_2}\)需要对两个变元都满足线性性,于是可以定义
    \begin{align*}
        \ip*{v_1}{v_2}=\ip*{\sum_{i=1}^m a_ie_i}{\sum_{j=1}^m b_je_j}=
        \sum_{i=1}^m \sum_{j=1}^m a_ib_j \ip*{e_i}{e_j}=\sum_{i=1}^m \sum_{j=1}^m a_ib_j\delta_{i,j}
    \end{align*}
    其中\(\delta_{i,j}\)是\textit{Kronecker}函数.下面验证内积\(\ip*{v_1}{v_2}\)的正定性.
    
    令\(v_2=v_1\),则\(\ip*{v_1}{v_1}=\sum_{j=1}^m a_ia_j\delta_{i,j}=\sum_{i=1}^m \abs*{a_i}^2 \geq 0\).
    
    且\(\sum_{i=1}^m \abs*{a_i}^2=0\)可推出\(a_1=\cdots=a_m=0\),从而\(v_1=0\),满足正定性,该内积定义合法.
    
    在这种内积定义下,\(\forall i,j=1,\cdots,n,\ip*{e_i}{e_j}=\delta_{i,j}\),即\(e_1,\cdots,e_m\)是一组规范正交基.
    
    于是\(M(T)\)为对角矩阵且对角线上元素均为实数,则\(T\)是自伴算子.
\end{proof}
% End: source/chapter_7/7.B.tex

