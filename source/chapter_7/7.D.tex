\section{7.D Singular Value Decomposition and Polar Decomposition}

\begin{lemma}[7.64*]\label{lem 7.64*}
    设\(V\)是有限维复内积空间且\(T \in L(V)\).则有

    a.\(T^*T \in L(V),TT^* \in L(W)\)是正算子. \quad b.\(\ker T^*T=\ker T\). \quad c.\(\operatorname{Im} T^*T=\operatorname{Im} T^*\).
\end{lemma}

\begin{proof}[证明a]
    由于\((T^*T)^*=T^*(T^*)^*=T^*T\),即\(T^*T\)为自伴算子.

    从而\(\forall v \in V,\ip*{T^*Tv}{v}=\ip*{Tv}{Tv}=\norm*{Tv}^2 \geq 0\),即\(T^*T\)为正算子.\(TT^*\)同理.
\end{proof}

\begin{proof}[证明b]
    设\(v \in \ker T^*T\),则\(\ip*{T^*Tv}{v}=\ip*{Tv}{Tv}=\norm*{Tv}^2=0\),即\(\ker T^*T \subseteq \ker T\).

    \(\ker T \subseteq \ker T^*T\)是显然的.结合之,即有\(\ker T^*T=\ker T\).    
\end{proof}

\begin{proof}[证明c]
    \(\operatorname{Im} T^*T=(\ker (T^*T)^*)^\bot=(\ker T^*T)^\bot=(\ker T)^\bot=\operatorname{Im} T^*\).
\end{proof}

\begin{theorem}[7.70*]\label{thm 7.70*} 奇异值分解 \:
    设\(V\)是有限维内积空间且\(T \in L(V)\),其奇异值为\(s_1,\cdots,s_n\).

    那么存在\(V\)的两组规范正交基\(e_1,\cdots,e_n\)和\(f_1,\cdots,f_n\),
    使得\(\forall v \in V,Tv=\sum_{i=1}^n s_i\ip*{v}{e_i}f_i\).
\end{theorem}

\begin{proof}
    \(T^*T\)作为自伴算子有一组规范正交特征基\(e_1,\cdots,e_n\)满足\(\forall i=1,\cdots,n,T^*Tv=s_i^2 v\).

    定义\(\forall i=1,\cdots,n,f_i\)为\(\dfrac{1}{s_i}Te_i\),下证\(f_1,\cdots,f_n\)为\(V\)的规范正交基.
    \begin{align*}
        \ip*{f_j}{f_k}=\dfrac{1}{s_js_k}\ip*{Te_j}{Te_k}=\dfrac{1}{s_js_k}\ip*{T^*Te_j}{e_k}
        =\dfrac{1}{s_js_k}\ip*{s_j^2 e_j}{e_k}=\dfrac{s_j}{s_k}\ip*{e_j}{e_k}=\delta_{j,k}
    \end{align*}
    其中\(\delta_{i,k}\)是\textit{Kronecker}函数.因此,\(f_1,\cdots,f_n\)确实为\(V\)的规范正交基.
    \begin{align*}
        Tv=T\sum_{i=1}^n \ip*{v}{e_i}e_i=\sum_{i=1}^n \ip*{v}{e_i}Te_i
        =\sum_{i=1}^n \ip*{v}{e_i}s_if_i=\sum_{i=1}^n s_i\ip*{v}{e_i}f_i
    \end{align*}
    因此\(M(T,(e_1,\cdots,e_n),(f_1,\cdots,f_n))=\mathrm{diag}(s_1,\cdots,s_n)\).
\end{proof}

\begin{theorem}[7.93*]\label{thm 7.93*} 极分解 \:
    设\(T \in L(V)\),则存在等距映射\(S \in L(V)\)使得\(T=S\sqrt{T^*T}\).
\end{theorem}

\begin{proof}
    沿用奇异值分解证明中的记号,设\(\forall v \in V,Tv=\sum_{i=1}^n s_i\ip*{v}{e_i}f_i\).

    定义\(S \in L(V)\)为\(Sv=\sum_{i=1}^n \ip*{v}{e_i}f_i\),下面验证\(S\)是等距映射.
    \begin{align*}
        \norm*{Sv}^2=\norm*{\sum_{i=1}^n \ip*{v}{e_i}f_i}^2=\sum_{i=1}^n \norm*{\ip*{v}{e_i}f_i}^2
        =\sum_{i=1}^n \norm*{\ip*{v}{e_i}}^2=\norm*{v}^2
    \end{align*}
    其中第二个等号来自定理6.25,第四个等号来自定理6.30.
    
    由\(T^*v=\sum_{i=1}^n s_i\ip*{v}{f_i}e_i\)得到\(T^*Tv=\sum_{i=1}^n s_i^2 \ip*{v}{e_i}e_i\),
    因而\(\sqrt{T^*T}v=\sum_{i=1}^n s_i\ip*{v}{e_i}e_i\).
    \begin{align*}
        S\sqrt{T^*T}v=S\sum_{i=1}^n s_i\ip*{v}{e_i}e_i=\sum_{i=1}^n s_i\ip*{v}{e_i}Se_i
        =\sum_{i=1}^n s_i\ip*{v}{e_i}f_i=Tv
    \end{align*}
    因此\(S\)就是所求的等距算子.
\end{proof}

\newpage

\begin{problem}[1]\label{7.D.1}
    给定\(u \ne 0,x \in V,Tv=\ip*{v}{u}x\).证明:\(\sqrt{T^*T}v=\dfrac{\norm*{x}}{\norm*{u}}\ip*{v}{u}u\).
\end{problem}

\begin{proof}
    根据\probref{7.A.15},\((T^*T)v=T^*(\ip*{v}{u}x)=\ip*{v}{u}T^*x=\norm*{x}^2\ip*{v}{u}u\).

    令\(Rv=\dfrac{\norm*{x}}{\norm*{u}}\ip*{v}{u}u\),验证\(R\)是\(T^*T\)的平方根即可.
        \begin{align*}
            R^2v=\dfrac{\norm*{x}}{\norm*{u}}\ip*{\dfrac{\norm*{x}}{\norm*{u}}\ip*{v}{u}u}{u}u
            =\dfrac{\norm*{x}}{\norm*{u}}\dfrac{\norm*{x}}{\norm*{u}}\ip*{v}{u}\norm*{u}^2u
            =\dfrac{\norm*{x}}{\norm*{u}}\ip*{v}{u}u=Tv
        \end{align*}
\end{proof}

\begin{problem}[2]\label{7.D.2}
    给出\(T \in L(C^2)\)满足\(0\)是其唯一特征值,但奇异值为\(0,5\).
\end{problem}

\begin{proof}
    令\(T(z_1,z_2)=(0,5z_1)\),则\(0\)是其唯一特征值.得\(T^*(z_1,z_2)=(5z_2,0)\),
    
    于是\(T^*T(z_1,z_2)=(25z_1,0)\),即\(T^*T(1,0)=(25,0),T^*T(0,1)=(0,0)\).
\end{proof}

\begin{problem}[3]\label{7.D.3}
    设\(T \in L(V)\).求证:存在等距算子\(S \in L(V)\)满足\(T=\sqrt{TT^*}S\).
\end{problem}

\begin{proof}
    利用极分解的结论,\(T=S\sqrt{T^*T}\),从而\(T^*=\sqrt{T^*T}S^*\).

    将\(T^*\)替换成\(T\),\(S^*\)替换成\(S\),即\(T=\sqrt{TT^*}S\).
\end{proof}

\begin{problem}[8]\label{7.D.8}
    设\(T,S,R \in L(V)\).\(S\)是等距算子.\(R\)是满足\(T=SR\)的正算子.
    
    求证:\(R=\sqrt{T^*T}\).
\end{problem}

\begin{proof}
    \(T^*=(SR)^*=R^*S^*\),故\(T^*T=R^*(S^*S)R=R^*R=R^2\),据定理7.35有\(R=\sqrt{T^*T}\).
\end{proof}

\begin{problem}[9]\label{7.D.9}
    设\(T \in L(V)\),证明:\(T\)是可逆算子当且仅当其极分解是唯一的.
\end{problem}

\begin{proof}
    根据引理7.64*,\(T\)是可逆算子当且仅当\(T^*T,\sqrt{T^*T}\)是可逆算子.

    必要性:设\(T=S_1\sqrt{T^*T}=S_2\sqrt{T^*T}\).由于\(\sqrt{T^*T}=\sqrt{T^*T}^*\),故\(T^*=\sqrt{T^*T}S_1^*\).
    
    于是\(T^*T=\sqrt{T^*T}\sqrt{T^*T}=\sqrt{T^*T}(S_1^*S_2)\sqrt{T^*T}\).
    
    \(\sqrt{T^*T}\)是可逆算子,故两侧的\(\sqrt{T^*T}\)可以被消去,得到\(S_1^*S_2=I\).
    
    由于\(S_1,S_2\)都是等距算子,故\(S_1^*=S^{-1},S_2^*=S^{-2},S_1=S_2\),反之亦然.
    
    充分性:设\(T\)不可逆,则\(\ker \sqrt{T^*T}=\ker T \ne \{0\}\),取其中任意向量\(v\).
    
    \(S\)作用于\(v\)的结果可以是任意的,从而导致了\(S\)的不唯一,矛盾.
\end{proof}

\begin{problem}[11]\label{7.D.11}
    设\(T \in L(V)\).证明:\(T\)和\(T^*\)有相同的奇异值.
\end{problem}

\begin{proof}
    由\(T^*T\)是自伴算子,\(E(\lambda,\sqrt{T^*T}) \ne \{0\} \Leftrightarrow E(\lambda,\sqrt{T^*T}^*)=E(\lambda,\sqrt{TT^*}) \ne \{0\}\).
    
    因此\(\lambda\)是\(T\)的奇异值和\(\lambda\)是\(T^*\)的奇异值是等价的.
\end{proof}

\begin{problem}[15]\label{7.D.15}
    设\(S \in L(V)\).求证:\(S\)是等距算子等价于\(S\)的所有奇异值均为\(1\).
\end{problem}

\begin{proof}
    充分性:\(\sqrt{S^*S}=\sqrt{S^{-1}S}=I\),显然其特征值均为\(1\),故\(S\)的所有奇异值均为\(1\).

    必要性:沿用定理7,70*证明中的记号,则\(Sv=\sum_{i=1}^n \ip*{v}{e_i}f_i\).
    
    沿用定理7.93*的证明即可得到\(\norm*{Sv}=v\).
\end{proof}

\newpage

\begin{problem}[17]\label{7.D.17}
    沿用定理7.70*证明的记号,设\(T \in L(V)\)满足\(Tv=\sum_{i=1}^n s_i\ip*{v}{e_i}f_i\),求证:

    a.\(T^*v=\sum_{i=1}^n s_i\ip*{v}{f_i}e_i\).
    
    b.\(T^*Tv=\sum_{i=1}^n s_i^2 \ip*{v}{e_i}e_i\). 
    
    c.\(\sqrt{T^*T}v=\sum_{i=1}^n s_i\ip*{v}{e_i}e_i\).
    
    d.若\(T\)可逆,则\(T^{-1}v=\sum_{i=1}^n s_i^{-1}\ip*{v}{f_i}e_i\).    
\end{problem}

\begin{proof}[证明a]
    设\(\forall v,w \in V\),利用\(\ip*{Tv}{w}=\ip*{v}{T^*w}\),有
    \begin{align*}
        \ip*{Tv}{w}&=\ip*{\sum_{i=1}^n s_i\ip*{v}{e_i}f_i}{w}=\sum_{i=1}^n s_i\ip*{v}{e_i}\ip*{f_i}{w} \\
        &=\sum_{i=1}^n \ip*{v}{\ol{s_i\ip*{f_i}{w}}e_i}=\sum_{i=1}^n \ip*{v}{s_i\ip*{w}{f_i}e_i}=\ip*{v}{T^*w}
    \end{align*}
    即\(T^*w=\sum_{i=1}^n s_i\ip*{w}{f_i}e_i\).
\end{proof}

\begin{proof}[证明b]
    \(T^*(Tv)=T^*\sum_{i=1}^n s_i \ip*{v}{e_i}f_i=\sum_{i=1}^n s_i \ip*{v}{e_i}T^*f_i=\sum_{i=1}^n s_i^2 \ip*{v}{e_i}e_i\).
\end{proof}

\begin{proof}[证明c]
    令\(Rv=\sum_{i=1}^n s_i\ip*{v}{e_i}e_i\),验证\(R\)是\(T^*T\)的平方根即可.
    \begin{align*}
        R^2v=\sum_{i=1}^n s_i\ip*{\sum_{i=1}^n s_i\ip*{v}{e_i}e_i}{e_i}e_i
        =\sum_{i=1}^n s_i (\sum_{i=1}^n s_i\ip*{v}{e_i}\ip*{e_i}{e_i})e_i
        =\sum_{i=1}^n s_i^2\ip*{v}{e_i}e_i
    \end{align*}
    容易验证\(R\)是正算子,故根据定理7.36,\(R=\sqrt{T^*T}\).
\end{proof}

\begin{proof}[证明d]
    令\(w=\sum_{i=1}^n s_i^{-1}\ip*{v}{f_i}e_i\),则
    \begin{align*}
        Tw=\sum_{i=1}^n s_i^{-1}\ip*{v}{f_i}Te_i=\sum_{i=1}^n s_i^{-1}\ip*{v}{f_i}s_if_i
        =\sum_{i=1}^n \ip*{v}{f_i}f_i=v
    \end{align*}
    因此\(Tw=v,w=T^{-1}v=\sum_{i=1}^n s_i^{-1}\ip*{v}{f_i}e_i\).
\end{proof}

\begin{proof}[18]\label{7.D.18}
    设\(T \in L(V)\).令\(s_{\min},s_{\max}\)分别指代\(T\)的最小和最大奇异值,\(\lambda\)是其特征值.求证:

    a.\(\forall v \in V,s_{\min}\norm*{v} \leq \norm*{Tv} \leq s_{\max}\norm*{v}\). \quad
    b.\(s_{\min} \leq \abs*{\lambda} \leq s_{\max}\).
\end{proof}

\begin{proof}[证明a]
    设\(Tv=\sum_{i=1}^n s_i\ip*{v}{e_i}f_i\),则
    \begin{align*}
        s_{\min}^2\norm*{v}^2w=s_{\min}^2 \sum_{i=1}^n \norm*{\ip*{v}{e_i}f_i}^2 
        \leq \sum_{i=1}^n \norm*{s_i\ip*{v}{e_i}f_i}^2 
        \leq s_{\max}^2 \sum_{i=1}^n \norm*{\ip*{v}{e_i}f_i}^2=s_{\max}^2\norm*{v}^2
    \end{align*}
    因此\(s_{\min}\norm*{v} \leq \norm*{Tv}=\norm*{\sum_{i=1}^n s_i\ip*{v}{e_i}f_i} \leq s_{\max}\norm*{v}\).
\end{proof}

\begin{proof}[证明b]
    设\(v\)是\(\lambda\)的特征向量,
    则\(s_{\min}\norm*{v}\leq \norm*{Tv}=\norm*{\lambda v}=\abs*{\lambda}\norm*{v} \leq s_{\max}\norm*{v}\).
    
    于是\(s_{\min} \leq \abs*{\lambda} \leq s_{\max}\).
\end{proof}
% End: source/chapter_7/7.D.tex

