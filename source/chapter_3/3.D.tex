\section{3.D Invertibility and Isomorphic Vector Spaces}

\begin{problem}[1]\label{3.D.1}
    设\(T \in L(U,V),S \in L(V,W)\)均可逆.

    求证:\(ST \in L(U,W)\)可逆,且\((ST)^{-1}=T^{-1}S^{-1}\).
\end{problem}

\begin{proof}
    先证明\(ST\)是单射变换.令\((ST)(u)=0\),则\(S(Tu)=0\).

    而\(\ker S=\{0\}\),故\(Tu=0\).由于\(\ker T=\{0\}\),故\(u=0\),得\(\ker ST=\{0\}\),证毕.
    
    再证明\(ST\)是满射变换.
    由于\(\operatorname{Im} S=W\),故对于\(\forall w \in W\),\(\exists v \in V\),使得\(Sv=w\).
    
    又由于\(\operatorname{Im} T=V\),故对于\(\forall v \in V\),\(\exists u \in U\),使得\(Tu=v\).
    
    因此,\(\forall w \in W\),\(\exists u \in U\),使得\((ST)u=S(Tu)=Sv=w\),即\(\operatorname{Im} ST=W\).
    
    可得\(ST\)是满射变换,因此\(ST\)是可逆变换,证毕.
    
    再证\((ST)^{-1}=T^{-1}S^{-1}\).两边同乘\(ST\),得
    \begin{align*}
        (ST)(T^{-1}S^{-1})=S(TT^{-1})S^{-1}=SS^{-1}=I=(ST)(ST)^{-1} 
    \end{align*}
\end{proof}

\begin{problem}[2]\label{3.D.2}
    设\(V\)是有限维向量空间且\(\dim V \geq 1\).
    
    求证:\(L(V)\)中所有不可逆算子不构成\(L(V)\)的子空间.    
\end{problem}

\begin{proof}
    设\(v_1,\cdots,v_m\)是\(V\)的一组基.定义
    \begin{align*}
        T_1v_1=0,T_1v_i=v_i,i=2,\cdots,m \quad
        T_2v_i=v_i,T_2v_m=0,i=1,\cdots,m-1
    \end{align*}
    很显然\(T_1,T_2\)都不是可逆变换,但是\(T_1+T_2\)是一个可逆变换,证毕.    
\end{proof}

\begin{problem}[3]\label{3.D.3}
    设\(V\)是有限维向量空间且\(U\)是\(V\)的一个子空间,\(S \in L(U,V)\).

    求证:\(S\)是单射变换的充要条件是\(\exists T \in L(V)\),使得\(\forall u \in U,Tu=Su\).        
\end{problem}

\begin{proof}
    必要性:由\(T\)为单射变换,而\(\forall u \in U,Tu=Su\),故\(S\)也是单射变换.
    
    充分性:设\(u_1,\cdots,u_m\)是\(U\)的一组基,扩充为\(V\)的基\(u_1,\cdots,u_m,v_1,\cdots,v_n\).
    
    由于\(S\)是单射变换,根据\probref{3.B.9},\(Su_1,\cdots,Su_m\)在\(V\)中线性无关.
    
    故存在线性无关的\(w_1,\cdots,w_n\), 使得\(Su_1,\cdots,Su_m,w_1,\cdots,w_n\)是\(V\)的一组基.
    
    定义线性算子\(T\)为\(Tu_i=Su_i,i=1,\cdots,m \quad Tv_j=w_j,j=1,\cdots,n\).
    
    很显然\(T\)是单射变换且\(T \in L(V)\),故\(T\)是可逆变换.下证\(Tu=Su\).
    \begin{align*}
        \forall u=\sum_{i=1}^m a_iu_i,Tu=T\sum_{i=1}^m a_iu_i
        =\sum_{i=1}^m a_iTu_i=\sum_{i=1}^m a_iSu_i=S\sum_{i=1}^m a_iu_i=Su 
    \end{align*}
\end{proof}

\newpage

\begin{problem}[4]\label{3.D.4}
    设\(W\)是有限维向量空间且\(T_1,T_2 \in L(V,W)\).

    求证:\(\ker T_1=\ker T_2\)的充要条件是存在可逆算子\(S \in L(W)\),使得\(T_1=ST_2\).
\end{problem}

\begin{proof}
    必要性:若\(S\)是可逆变换,那么
    \begin{align*}
        \forall v \in V, T_2v=0 \Longleftrightarrow (ST_2)v=0 \Longleftrightarrow T_1v=0
        \Longrightarrow \ker T_1=\ker T_2
    \end{align*}
    充分性:记\(\ker T_1=\ker T_2=K\).定义\(\varphi_i:V/K \rightarrow \operatorname{Im} T_i\)为\(\varphi_i(v+K)=T_iv\).

    定义\(S_0:\operatorname{Im} T_2 \rightarrow \operatorname{Im} T_1\)为\(S_0=\varphi_1 \circ \varphi_2^{-1}\).根据同态定理,\(S_0\)是可逆变换.

    由于\(\operatorname{Im} T_1, \operatorname{Im} T_2 \subseteq W\),分别给出它们相对于\(W\)的补空间\(U_1, U_2\)满足\(\dim U_1=\dim U_2\).

    因此存在\(S':U_2 \rightarrow U_1\)为可逆变换.定义\(S:W \rightarrow W\)为
    \begin{align*}
        &\forall w=w_1+w_2 \in W, w_1 \in \operatorname{Im} T_2, w_2 \in U_2, Sw=S(w_1+w_2)=S_0w_1+S'w_2 \\
        &\forall v \in V, T_1v=\varphi_1(v+K)=(S_0 \circ \varphi_2)(v+K)=S(T_2v) \qedhere
    \end{align*}
\end{proof}

\begin{problem}[5]\label{3.D.5}
    设\(V\)是有限维向量空间且\(T_1,T_2 \in L(V,W)\).

    求证:\(\operatorname{Im} T_1=\operatorname{Im} T_2\)的充要条件是存在可逆算子\(S \in L(V)\),使得\(T_1=T_2S\).    
\end{problem}

\begin{proof}
    必要性:显然\(\operatorname{Im} T_1=T_1(V)=T_2S(V)=T_2(S(V))=T_2(V)=\operatorname{Im} T_2\).
    
    充分性:考虑对偶变换\(T'_1, T'_2: W' \rightarrow V'\).由\(\operatorname{Im} T_1=\operatorname{Im} T_2\),故\(\ker T'_1=\ker T'_2\).

    根据\probref{3.D.4},存在\(S': V' \rightarrow V'\),使得\(T'_1=S'T'_2\).

    两边取对偶,有\(T''_1=(S'T'_2)'=T''_2S''\),即\(T_1=T_2S\).
\end{proof}

\begin{comment}
\begin{problem}[6]\label{3.D.6}
    设\(V\)和\(W\)是有限维向量空间且\(T_1,T_2 \in L(V,W)\).

    求证:\(\dim \ker T_1=\dim \ker T_2\)等价于存在可逆算子\(R \in L(V),S \in L(W)\),使得\(T_1=ST_2R\).    
\end{problem}

\begin{proof}
    必要性:\(T_1=ST_2R \Rightarrow S^{-1}T_1=T_2R\).

    根据\probref{3.D.4},\(\ker T_1=\ker T_2R\).根据\probref{3.D.5},\(\operatorname{Im} S^{-1}T_1=\operatorname{Im} T_2\).
    \begin{align*}
        \dim \ker T_2R=\dim V-\dim \operatorname{Im} T_2R=\dim V-\dim \operatorname{Im} T_2=\dim \ker T_2
    \end{align*}
    充分性:令\(v_1^\alpha,\cdots,v_m^\alpha\)和\(v_1^\beta,\cdots,v_m^\beta\)分别为\(\ker T_1\)和\(\ker T_2\)的一组基.
    
    将\(v_1^\alpha,\cdots,v_m^\alpha\)和\(v_1^\beta,\cdots,v_m^\beta\)分别补充成\(V\)的一组基
    
    \(v_1^\alpha,\cdots,v_m^\alpha,u_1^\alpha,\cdots,u_n^\alpha\),其中\(u_1^\alpha,\cdots,u_n^\alpha\)线性无关;
    
    \(v_1^\beta,\cdots,v_m^\beta,u_1^\beta,\cdots,u_n^\beta\),其中\(u_1^\beta,\cdots,u_n^\beta\)线性无关.
    
    根据\probref{3.B.9}和\probref{3.B.12},\(T_1u_1^\alpha,\cdots,T_1u_n^\alpha\)是\(\operatorname{Im} T_1\)的一组基,
    \(T_2u_1^\beta,\cdots,T_2u_n^\beta\)是\(\operatorname{Im} T_2\)的一组基.
    
    由于\(\operatorname{Im} T_1,\operatorname{Im} T_2 \subseteq W\),
    将\(T_1u_1^\alpha,\cdots,T_1u_n^\alpha\)和\(T_2u_1^\beta,\cdots,T_2u_n^\beta\)分别补充成\(W\)的一组基
    
    \(T_1u_1^\alpha,\cdots,T_1u_n^\alpha,w_1^\alpha,\cdots,w_p^\alpha\),其中\(w_1^\alpha,\cdots,w_p^\alpha\)线性无关;
    
    \(T_2u_1^\beta,\cdots,T_2u_n^\beta,w_1^\beta,\cdots,w_p^\beta\),其中\(w_1^\beta,\cdots,w_p^\beta\)线性无关.
    
    现在定义线性变换\(R\)和\(S\).令
    \begin{align*}
        Rv_i^\alpha=v_i^\beta,i=1,\cdots,m &\quad Ru_j^\alpha=u_j^\beta,j=1,\cdots,n \\
        S(T_2u_j^\beta)=T_1u_j^\alpha,j=1,\cdots,m &\quad Sw_k^\beta=w_k^\alpha,k=1,\cdots,p
    \end{align*}
    参考\probref{3.D.4}和\probref{3.D.5}的证明,显然\(R\)和\(S\)都是可逆变换.
    
    下证\(T_1=ST_2R\).\(\forall v=\sum_{i=1}^m a_iv_i^\alpha+\sum_{j=1}^n b_ju_j^\alpha\),
    \begin{align*}
        (ST_2R)v&=(ST_2R)(\sum_{i=1}^m a_iv_i^\alpha+\sum_{j=1}^n b_ju_j^\alpha)
                =(ST_2)(\sum_{i=1}^m a_iRv_i^\alpha+\sum_{j=1}^n b_jRu_j^\alpha) \\
                &=(ST_2)(\sum_{i=1}^m a_iv_i^\beta+\sum_{j=1}^n b_ju_j^\beta)
                =S(\sum_{i=1}^m a_iT_2v_i^\beta+\sum_{j=1}^n b_jT_2u_j^\beta) \\
                &=\sum_{j=1}^n b_jS(T_2u_j^\beta)=\sum_{j=1}^n b_jT_1u_j^\alpha
                =T_1(\sum_{i=1}^m a_iv_i^\alpha+\sum_{j=1}^n b_ju_j^\alpha)=T_1v
    \end{align*}
    故\(\forall v \in V,T_1v=(ST_2R)v\),即\(T_1=ST_2R\),证毕.
\end{proof}
\end{comment}

\begin{problem}[8]\label{3.D.8}
    设\(V\)是有限维向量空间且\(T \in L(V,W)\)是一个满射变换.

    求证:存在\(V\)的一个子空间\(U\),使得\(T|_U\)是一个可逆变换.    
\end{problem}

\begin{proof}
    令\(w_1,\cdots,w_m\)为\(\operatorname{Im} T\)的一组基.找到\(v_1,\cdots,v_m\),使得\(\forall i=1,\cdots,m,T_1v_i=w_i\).

    根据习题3.A.4,\(v_1,\cdots,v_m\)线性无关.令\(U=\operatorname{span} (v_1,\cdots,v_m)\).
    
    下证\(T|_U\)是一个可逆变换.先证\(\ker T|_U=\{0\}\).令\(Tv=T\sum_{i=1}^m a_iv_i=\sum_{i=1}^m a_iw_i=0\).
    
    由于\(w_1,\cdots,w_m\)线性无关,故\(a_1=\cdots=a_m=0\),从而\(\ker T|_U=\{0\}\),证毕.
    
    又因为\(\operatorname{Im} T|_U=\operatorname{Im} T=\operatorname{span} (w_1,\cdots,w_m)\),故\(T|_U\)是满射变换.综上,\(T|_U\)是可逆变换.
\end{proof}

\newpage

\begin{problem}[14]
    设\(v_1^,\cdots,v_n\)是\(V\)的基.定义\(T \in L(V,F^{n,1})\)为\(Tv=M(v)\).求证:\(T\)是可逆变换.    
\end{problem}

\begin{proof}
    \(\forall v \in V,\exists c_1,\cdots,c_n \in F\),使得\(v=\sum_{i=1}^n c_iv_i\).
    从而\(T\sum_{i=1}^n c_iv_i=(c_1,\cdots,c_n)^T\).

    下证\(T\)是单射变换.令\(M(v)=0\),则\(c_1=\cdots=c_n=0\),即\(v=0\),从而\(T\)是单射变换.

    再证\(T\)是满射变换.\(\forall c_i \in F,\exists v \in V\),使得\(v=\sum_{i=1}^n c_iv_i\).因此\(T\)是满射变换,证毕.
\end{proof}

\begin{problem}[15]\label{3.D.15}
    证明:若\(T \in L(F^{n,1},F^{m,1})\),则存在一个\(m \times n\)矩阵\(A\),
    
    使得\(\forall x \in F^{n,1},Tx=Ax\).
\end{problem}

\begin{proof}
    使用\(F^{n,1}\)和\(F^{m,1}\)的标准基\(e_1,\cdot,e_n\)和\(f_1,\cdots,f_m\).
    \begin{align*}
        M(T,(e_1,\cdots,e_n),(f_1,\cdots,f_m))=
            \begin{pmatrix}
                A_{1,1} & \cdots & A_{1,n}  \\
                \vdots  & \ddots & \vdots   \\
                A_{m,1} & \cdots & A_{m,n}
            \end{pmatrix}
        =A
    \end{align*}
    因此总是存在这样的\(A_{i,j}\),证毕.
\end{proof}

\begin{problem}[17]\label{3.D.17}
    设\(V\)是有限维向量空间,\(\varepsilon\)是\(L(V)\)的一个子空间,

    其对所有\(S \in L(V)\)和\(T \in \varepsilon\)满足\(ST \in \varepsilon\)和\(TS \in \varepsilon\).
    求证:\(\varepsilon=\{0\}\)或\(\varepsilon=L(V)\).    
\end{problem}

\begin{proof}
    设\(e_1,\cdots,e_n\)是\(V\)的一组基.当\(\varepsilon =\{0\}\)时,\(ST=TS=0 \in \varepsilon\)是显然的;

    若\(\varepsilon \ne \{0\}\),则\(\exists T \in \varepsilon, T \ne 0\).
    从而\(\exists s,t \in \{1,\cdots,n\},Te_s=\sum_{i=1}^n a_ie_i ,Te_s \ne 0, a_t \ne 0\).
    
    现在定义\(L(V)\)的一组基.令\(E_{i,j}e_k=\delta_{i,k}e_j\),其中\(\delta_{i,k}\)是\textit{Kronecker}函数.
    
    从而当\(E_{i,j}\)中\(i,j\)分别取遍\(\{1,\cdots,n\}\)中每一个值时,\(E_{i,j}\)成为\(L(V)\)的一组基.
    \begin{align*}
        E_{t,i}TE_{i,s}e_j=E_{t,i}T(\delta_{i,j} e_s)=\delta_{i,j}E_{t,i}a_te_t=a_t\delta_{i,j}e_i
    \end{align*}
    由于\(E_{t,i}TE_{i,s} \in \varepsilon\)且\(\varepsilon\)是\(L(V)\)的子空间,
    故\(\sum_{i=1}^n E_{t,i}TE_{i,s}e_j=a_t \sum_{i=1}^n \delta_{i,j}e_i\).
    
    由于只有\(i=j\)时,\(\delta_{i,j}=1\),因此除\(i=j\)的其它项被消去,得\(\sum_{i=1}^n E_{t,i}TE_{i,s} e_j=a_te_j\).
    
    即\(\sum_{i=1}^n E_{t,i} T E_{i,s} =a_t I\).因此\(a_t I \in \varepsilon \Rightarrow I \in \varepsilon\),
    故而\(\forall S \in L(V),S=SI \in \varepsilon\),即\(\varepsilon=L(V)\).
\end{proof}

\begin{comment}
\begin{problem}[19]\label{3.D.19}
    设\(T \in \mathcal{L}(P(\mathbb{R}))\)满足\(T\)是单射变换,且\(\forall p \ne 0 \in P(R)\)满足\(\operatorname{deg} Tp \leq \operatorname{deg} p\).

    a.证明:\(T\)是满射变换.

    b.证明:对于任意非零多项式\(p \in P(R)\)都有\(\operatorname{deg} Tp=\operatorname{deg} p\).
\end{problem}

\begin{proof}[证明a]
    由于\(\operatorname{deg} Tp \leq \operatorname{deg} p\),故\(M(T,(1, x, \dots))\)是上三角矩阵.

    由于\(T\)是单射变换,故对角线元素均不为零,即\(\operatorname{Im} T=P(\mathbb{R})\),证毕.
\end{proof}

\begin{proof}[证明b]
    使用反证法,若存在\(p\)使得\(\operatorname{deg} Tp<\operatorname{deg} p\),那么该列对角线元素为零,矛盾.
\end{proof}
\end{comment}
% End: source/chapter_3/3.D.tex

