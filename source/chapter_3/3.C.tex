\section{3.C Matrices}

\begin{problem}[6]\label{3.C.6}
    设\(V\)和\(W\)都是有限维向量空间且\(T \in L(V,W)\).

    求证:\(\dim \Img T=1\)等价于分别存在\(V\)和\(W\)的一组基,使得\(M(T)_{i,j}=1\).    
\end{problem}

\begin{proof}
    必要性:设\(v_1,\cdots,v_m\)和\(w_1,\cdots,w_n\)分别为\(V\)和\(W\)的一组基,

    且这两组基可以使得\(M(T)\)中所有元素均为1.
    
    从而有\(\forall i=1,\cdots,m,Tv_i=\sum_{i=1}^n w_i\),即\(\Img T=\altspan (\sum_{i=1}^m w_i),\dim \Img T=1\),证毕.
    
    充分性:设\(\mu_1,\cdots,\mu_m\)为\(V\)任意的一组基.
    
    由于\(\dim \Img T=1\),不妨设\(\Img T=\altspan (T \mu_1)\),即\(T \mu_2=\cdots=T \mu_m=0\).
    
    则一定存在线性无关的\(w_2,\cdots,w_n\),使得\(T \mu_1,w_2,\cdots,w_n\)是\(W\)的一组基.
    
    令\(w_1=T \mu_1-\sum_{i=2}^m w_i\),则\(w_1,w_2,\cdots,w_n\)也是\(W\)的一组基.
    
    再令\(v_1=\mu_1,v_i=\mu_i+\mu_1,i=2,\cdots,m\),因而\(Tv_i=T(\mu_1+\mu_i)=T \mu_1=\sum_{i=1}^m w_i\).
    
    从而\(v_1,\cdots,v_m\)和\(w_1,\cdots,w_m\)是满足条件的基.    
\end{proof}

\begin{problem}[7]\label{3.B.7}
    已知\(S,T \in L(V,W)\),求证:\(M(S+T)=M(S)+M(T)\).
\end{problem}

\begin{proof}
    设\(v_1,\cdots,v_m\)和\(w_1,\cdots,w_n\)分别为\(V\)和\(W\)的一组基,并令\(M(S)=A,M(T)=C\).
    \begin{align*}
        &Sv_j=\sum_{i=1}^m A_{i,j}w_i,Tv_j=\sum_{i=1}^m C_{i,j}w_i \quad
        Sv_j+Tv_j=\sum_{i=1}^m (A_{i,j}+C_{i,j})w_i=(S+T)v_j \\
        &\forall j=1,\cdots,m,i=1,\cdots,n,(M(S+T))_{i,j}=M(S)_{i,j}+M(T)_{i,j}
    \end{align*}
    即\(M(S+T)=M(S)+M(T)\),证毕.
\end{proof}

\begin{comment}
    \begin{problem}[13]\label{3.C.13}
        证明矩阵加法和乘法的分配律成立.
    \end{problem}

    \begin{proof}
        即证明\(A(B+C)=AB+AC,(D+E)F=DF+EF\).

        设\(A,D,E\)是\(m \times n\)矩阵,\(B,C,F\)是\(n \times p\)矩阵.

        \begin{enumerate}
            \item \(A(B+C)=AB+AC\).
            \begin{align*}
                &(A(B+C))_{i,j}=\sum_{k=1}^n A_{i,k}(B+C)_{k,j}=\sum_{k=1}^n A_{i,k}(B_{k,j}+C_{k,j}) \\
                &(AB+AC)_{i,j}=(AB)_{i,j}+(AC)_{i,j}=\sum_{k=1}^n A_{i,k}B_{k,j}+\sum_{k=1}^n A_{i,k}C_{k,j} \\
                &(A(B+C))_{i,j}=(AB+AC)_{i,j} \Rightarrow A(B+C)=AB+AC \\
            \end{align*}
            \item \((D+E)F=DF+EF\).
            \begin{align*}
                &((D+E)F)_{i,j}=\sum_{k=1}^n (D+E)_{i,k}F_{k,j}=\sum_{k=1}^n (D_{i,k}+E_{i,k})F_{k,j} \\
                &(DF+EF)_{i,j}=(DF)_{i,j}+(EF)_{i,j}=\sum_{k=1}^n D_{i,k}F_{k,j}+\sum_{k=1}^n E_{i,k}F_{k,j} \\
                &((D+E)F)_{i,j}=(DF+EF)_{i,j} \Rightarrow (D+E)F=DF+EF
            \end{align*}
        \end{enumerate}
            
    \end{proof}

    \begin{problem}[15]\label{3.B.15}
        设\(A\)是一个\(n \times n\)矩阵,且\(1 \leq i,j \leq n\).
        求证:\(A^3_{i,j}=\sum_{k=1}^n \sum_{p=1}^n A_{i,k}A_{k,p}A_{p,j}\).    
    \end{problem}

    \begin{proof}
    \begin{align*}
        A^3_{i,j}=\sum_{p=1}^n A^2_{i,p}A_{p,j}
        =\sum_{p=1}^n (\sum_{k=1}^n A_{i,k}A_{k,p})A_{p,j}
        =\sum_{k=1}^n \sum_{p=1}^n A_{i,k}A_{k,p}A_{p,j}
    \end{align*}
    \end{proof}
\end{comment}
% End: source/chapter_3/3.C.tex

