\section{3.B Null spaces and Ranges}

\begin{problem}[3]\label{3.B.3}
    设\(v_1,\cdots,v_m \in V\),定义\(T \in L(F^m,V)\)为\(T(z_1,\cdots,z_m)=\sum_{i=1}^m z_iv_i\).

    a.若\(V=\altspan (v_1,\cdots,v_m)\),则\(T\)具有怎样的性质?

    b.若\(v_1,\cdots,v_m\)线性无关,则\(T\)具有怎样的性质?
\end{problem}

\begin{proof}[证明a]
    由于\(\Img T=\altspan (v_1,\cdots,v_m)=V\),故\(T\)满射.
\end{proof}

\begin{proof}[证明b]
    令\(T(z_1,\cdots,z_m)=\sum_{i=1}^m z_iv_i=0\).

    由于\(v_1,\cdots,v_m\)线性无关,故\(z_1=\cdots=z_m=0\),即\(\ker T=\{0\}\),从而\(T\)单射.    
\end{proof}

\begin{problem}[9]\label{3.B.9}
    设\(T\in L(V,W)\)是单射的,且\(v_1,\cdots,v_m\)是\(V\)中一列线性无关的向量.

    求证:\(Tv_1,\cdots,Tv_m\)在\(W\)中线性无关.    
\end{problem}

\begin{proof}
    令\(\sum_{i=1}^m a_iTv_i=T\sum_{i=1}^m a_iv_i=0\),由于\(T\)是单射变换,故\(\sum_{i=1}^m a_iv_i=0\).

    而\(v_1,\cdots,v_m\)线性无关,则\(a_1=\cdots=a_m=0\),证毕.    
\end{proof}

\begin{problem}[10]\label{3.B.10}
    设\(V=\altspan (v_1,\cdots,v_m)\)且\(T\in L(V,W)\),求证:\(\Img T=\altspan (Tv_1,\cdots,Tv_m)\).
\end{problem}

\begin{proof}
    \(\Img T=\{Tv|v\in V\}=\{Tv|v\in \altspan(v_1,\cdots,v_m)\}\).

    \(\forall u=T\sum_{i=1}^m a_iv_i \in \Img T,u=\sum_{i=1}^m a_iTv_i \in \altspan(Tv_1,\cdots,Tv_m)\);

    \(\forall u=\sum_{i=1}^m a_iTv_i \in \altspan(Tv_1,\cdots,Tv_m),u=T\sum_{i=1}^m a_iv_i \in \Img T\).

    因而有\(\Img T \subseteq \altspan(Tv_1,\cdots,Tv_m),\altspan(Tv_1,\cdots,Tv_m) \subseteq \Img T\).

    即\(\Img T=\altspan (Tv_1,\cdots,Tv_m)\),证毕.
\end{proof}

\begin{problem}[12]\label{3.B.12}
    设\(V\)是有限维向量空间且\(T\in L(V,W)\).

    求证:存在\(V\)的一个子空间\(U\),满足\(U \cap \ker T=\{0\}\)且\(\Img T=\{Tu|u\in U\}\).    
\end{problem}

\begin{proof}
    令\(v_1,\cdots,v_m\)为\(\ker T\)的一组基,故存在线性无关的\(u_1,\cdots,u_n\),

    使得\(v_1,\cdots,v_m,u_1,\cdots,u_n\)为\(V\)的一组基.
    
    令\(U=\altspan (u_1,\cdots,u_n)\),则\(U\cap \ker T=\{0\}\)显然成立,下证\(\Img T=\{Tu|u\in U\}\).
    
    令\(\forall w\in V,\exists u\in U,v\in \ker T, w=u+v\),从而
    \begin{align*}
        \Img T=\{Tw|w\in V\}=\{T(u+v)|u\in U,v\in \ker T\}=\{Tu|u\in U\}
    \end{align*}
    从而\(\Img T=\{Tu|u\in U\}\),证毕.    
\end{proof}

\newpage

\begin{problem}[17]\label{3.B.17}
    设\(V\)和\(W\)是有限维向量空间.
    
    求证:\(\dim V \leq \dim W\)等价于存在单射的\(T\in L(V,W)\).
\end{problem}

\begin{proof}
    必要性:若\(T\in L(V,W)\)为单射,则\(\dim V=\dim \Img T \leq \dim W\),证毕.
    
    充分性:设\(v_1,\cdots,v_m\)和\(w_1,\cdots,w_n\)分别为\(V\)和\(W\)的一组基.
    
    有\(\dim V=m\leq n=\dim W\),从而定义\(Tv_i=w_i,i=1,\cdots,m\).
    
    定理3.5保证了线性变换\(T\)的存在性,下证\(T\)是单射的.
    
    令\(T\sum_{i=1}^m a_iv_i=\sum_{i=1}^m a_iTv_i=\sum_{i=1}^m a_iw_i=0\),从而\(a_1=\cdots=a_m=0\),证毕.    
\end{proof}

\begin{problem}[19]\label{3.B.19}
    设\(V\)和\(W\)是有限维的,\(U\)是\(V\)的一个子空间.

    求证:当且仅当\(\dim U \geq \dim V-\dim W\),存在\(T\in L(V,W)\),满足\(\ker T=U\).
\end{problem}

\begin{proof}
    必要性:原式等价于
    \begin{align*}
        \dim U=\dim \ker T=\dim V-\dim \Img T \geq \dim V-\dim W,\dim \Img T \leq \dim W
    \end{align*}
    充分性:设\(v_1,\cdots,v_m\)是\(U\)的一组基.由于\(U\)是\(V\)的一个子空间,

    故存在线性无关的\(v_{m+1},\cdots,v_{m+n}\)使得\(v_1,\cdots,v_{m+n}\)是\(V\)的一组基.

    从而原式等价于\(m\geq (m+n)-\dim W\),即\(\dim W\geq n\).令
    \begin{align*}
        Tv_i=0,i=1,\cdots,m \quad Tv_i=w_i,i=m+1,\cdots,m+n
    \end{align*}
    \(T\)满足\(\ker T=U\),同时\(\Img T=\altspan (Tv_{m+1},\cdots,Tv_{m+n})\),满足题意.
\end{proof}

\begin{problem}[20]\label{3.B.20}
    设\(V\)是有限维向量空间且\(T\in L(V,W)\).

    求证:\(T\)是单射变换等价于存在\(S \in L(W,V)\)满足\(ST=I_V\).
\end{problem}

\begin{proof}
    必要性:利用反证法.设\(T\)不是单射变换,
    即\(\exists v_\alpha,v_\beta \in V\)满足\(Tv_\alpha=Tv_\beta\)且\(v_\alpha \ne v_\beta\).
    
    \(\exists S,ST=I\),因而有\(v_\alpha=S(Tv_\alpha)=S(Tv_\beta)=v_\beta\).构成矛盾,即\(T\)为单射变换,证毕.
    
    充分性:设\(v_1,\cdots,v_m\)是\(V\)的一组基,显然\(Tv_1,\cdots,Tv_m\)线性无关.
    
    因而存在\(w_1,\cdots,w_n\),使得\(Tv_1,\cdots,Tv_m,w_1,\cdots,w_n\)是\(W\)的一组基.定义
    \begin{align*}
        S(Tv_i)=v_i,i=1,\cdots,m \quad Sw_j=0,j=1,\cdots,n
    \end{align*}
    定理3.5保证了线性变换\(S\)的存在性.下证\(ST=I\).
    \begin{align*}
        \forall v=\sum_{i=1}^m a_iv_i\in V,(ST)v=S(\sum_{i=1}^m a_iTv_i)
        =\sum_{i=1}^m a_iS(Tv_i)=\sum_{i=1}^m a_iv_i=v
    \end{align*}
\end{proof}

\newpage

\begin{problem}[21]\label{3.B.21}
    设\(V\)是有限维向量空间且\(T\in L(V,W)\).

    求证:\(T\)是满射变换等价于存在\(S\in L(W,V)\)满足\(TS=I_W\).    
\end{problem}

\begin{proof}
    必要性:利用反证法.设\(T\)不是满射变换,即\(\exists w\in W,w\notin \Img T\).

    \(\exists S,TS=I\),从而\(T(Sw)=w,w\in \Img T\),矛盾,必要性得证.
    
    充分性:设\(w_1,\cdots,w_m\)是\(W\)的一组基.由于\(T\)是满射变换,\(\Img T=\altspan (w_1,\cdots,w_m)\).
    
    设\(v_1,\cdots,v_m\in V,V=\altspan (v_1,\cdots,v_m,u_1,\cdots,u_n)\),则定义
    \begin{align*}
        Sw_i=v_i,i=1,\cdots,m \quad
        Tv_i=w_i,i=1,\cdots,m \quad Tu_j=0,j=1,\cdots,n
    \end{align*}
    定理3.5保证了线性变换\(S\)的存在性.下证\(TS=I\).
    \begin{align*}
        \forall w=\sum_{i=1}^m a_iw_i\in W,(TS)w=T(\sum_{i=1}^m a_iSw_i) 
        =\sum_{i=1}^m a_iTv_i=\sum_{i=1}^m a_iw_i=w
    \end{align*}
\end{proof}

\begin{problem}[24]\label{3.B.24}
    设\(W\)是有限维向量空间且\(T_1,T_2 \in L(V,W)\).

    求证:\(\ker T_1 \subseteq \ker T_2\)的充要条件是\(\exists S \in L(W)\),使得\(T_2=ST_1\).        
\end{problem}

\begin{proof}
    必要性:设\(\forall v \in \ker T_1\),有\(T_2v=ST_1v=0\).

    故\(\forall v \in \ker T_1, v \in \ker T_2\),即\(\ker T_1 \subseteq \ker T_2\),必要性得证.
    
    充分性:\(W\)是有限维向量空间且\(\Img T_1 \subseteq W\),故\(\Img T_1\)也是有限维向量空间.
    
    令\(T_1v_1,\cdots,T_1v_m\)是\(\Img T_1\)的一组基且\(\sum_{i=1}^m a_iv_i=0\),根据习题3.A.4,
    
    有\(\sum_{i=1}^m a_iT_1v_i=T_1(\sum_{i=1}^m a_iv_i)=0\),得\(a_1=\cdots=a_m=0\),从而\(v_1,\cdots,v_m\)线性无关.
    
    令\(K=\altspan (v_1,\cdots,v_m)\),从而\(V=K \oplus \ker T\).现在定义线性变换\(S\).
    
    由于\(T_1v_1,\cdots,T_1v_m\)线性无关,故将其补充为\(W\)的一组基\(T_1v_1,\cdots,T_1v_m,w_1,\cdots,w_n\).
    \begin{align*}
        S(T_1v_i)=T_2v_i,i=1,\cdots,m \quad Sw_j=0,j=1,\cdots,n
    \end{align*}
    定理3.5保证了线性变换\(S\)的存在性.下证\(ST_1=T_2\).
    
    由于\(\forall v \in V, v=v_0+\sum_{i=1}^m a_iv_i\),其中\(v_0 \in \ker T_1\),故
    \begin{align*}
        S(T_1v)=S(T_1v_0)+\sum_{i=1}^m a_iS(T_1v_i)
        =\sum_{i=1}^m a_iT_2v_i=T_2v_0+\sum_{i=1}^m a_iT_2v_i=T_2v
    \end{align*}
    由\(\ker T_1 \subseteq \ker T_2\),上式成立.因此\(\forall v \in V, ST_1v=T_2v\),证毕.
\end{proof}

\newpage

\begin{problem}[25]\label{3.B.25}
    设\(V\)是有限维向量空间且\(T_1,T_2 \in L(V,W)\).

    求证:\(\Img T_1 \subseteq \Img T_2\)的充要条件是\(\exists S \in L(V)\),使得\(T_1=T_2S\).    
\end{problem}

\begin{proof}
    必要性:设\(\forall v \in V\),有\(T_1v=T_2Sv \in \Img T_2\).

    故\(\forall v \in \Img T_1, v \in \Img T_2\),即\(\Img T_1 \subseteq \Img T_2\),必要性得证. 
    
    充分性:设\(u_1,\cdots,u_m\)是\(V\)的一组基,从而\(\Img T_1=\altspan(u_1,\cdots,u_m) \subseteq \Img T_2\).
    
    因此\(\exists v_1,\cdots,v_m \in V\),使得\(T_1u_i=T_2v_i,i=1,\cdots,m\).定义\(S\)为\(Su_i=v_i,i=1,\cdots,m\).
    \begin{align*}
        \forall u=\sum_{i=1}^m a_iu_i \in V,T_2S\sum_{i=1}^m a_iu_i=T_2\sum_{i=1}^m a_i(Su_i)
        =\sum_{i=1}^m a_iT_2v_i=\sum_{i=1}^m a_iT_1u_i=T_1\sum_{i=1}^m a_iu_i
    \end{align*}
    即\(\forall u \in V , T_1=T_2S\),充分性得证.    
\end{proof}

\begin{comment}
    \begin{problem}[26]\label{3.B.26}
        设\(D \in L(P(R))\)满足\(\forall p \in P(R),\altdeg Dp=\altdeg p-1\).求证:\(D\)是满射变换.
    \end{problem}

    \begin{proof}
        根据\probref{3.B.10},命题等价于
        \begin{align*}
            \altspan (D(x),D(x^2),\cdots)=\Img D=P(R)=\altspan (1,x,\cdots)
        \end{align*}
        根据\probref{2.C.10},由于\(\altdeg Dp=\altdeg p-1\),

        故\(\altspan (D(x),D(x^2),\cdots)=\altspan (1,x,\cdots)\)成立.        
    \end{proof}
\end{comment}

\begin{problem}[28]\label{3.B.28}
    设\(T \in L(V,W)\),且\(w_1,\cdots,w_m\)是\(\Img T\)的一组基.

    求证:\(\exists \varphi_1,\cdots,\varphi_m \in L(V,F)\),
    故\(\forall v \in V\)均满足\(Tv=\sum_{i=1}^m{\varphi_i(v)w_i}\).
\end{problem}

\begin{proof}
    由于\(w_1,\cdots,w_m\)是\(\Img T\)的一组基,
    故\(\forall Tv \in \Img T,\exists a_i \in F,Tv=\sum_{i=1}^m a_iw_i\).
    
    因此定义\(\varphi_i(v)=a_i\)即可,下证\(\forall i=1,\cdots,m,\varphi_i \in L(V,F)\).
    
    即证\(\forall i=1,\cdots,m,\forall v_\alpha,v_\beta \in V\),
    \(\varphi_i(\lambda v_\alpha+\mu v_\beta)=\lambda \varphi_i(v_\alpha)+\mu \varphi_i(v_\beta)\).
    \begin{align*}
        T(\lambda v_\alpha+\mu v_\beta)=\sum_{i=1}^m \lambda \varphi_i(v_\alpha)w_i+\sum_{i=1}^m \mu \varphi_i(v_\beta)w_i
        =\sum_{i=1}^m \varphi_i(\lambda v_\alpha+\mu v_\beta)w_i=\lambda Tv_\alpha+\mu Tv_\beta
    \end{align*}
    综上,\(\forall i=1,\cdots,m,\varphi_i \in L(V,F)\).    
\end{proof}

\begin{problem}[29]\label{3.B.29}
    设\(\varphi \in L(V,F)\).设\(u \in V \notin \ker \varphi\).求证:\(V=\ker \varphi \oplus \{au|a \in F\}\).
\end{problem}

\begin{proof}
    根据\probref{3.B.12},存在\(V\)的一个子空间\(K\),使得\(V=\ker \varphi \oplus K\)且\(\Img \varphi=\{\varphi (v)|v \in K\}\).
    
    因此只需证明\(K=\{au|a \in F\}\)即可.
    
    \(\forall v \in K,\varphi(v) \in F\).而\(\varphi(u) \ne 0 \in F\),
    故\(\exists a \in F\),使得\(\varphi(v)=a \varphi(u)=\varphi(au),a \ne 0\).

    所以\(\Img \varphi=\{\varphi (au)|a \in F\}\),即\(K=\{au|a \in F\}\),证毕.    
\end{proof}

\begin{problem}[30]\label{3.B.30}
    设\(\varphi_1,\varphi_2 \in L(V,F)\)且\(\ker \varphi_1=\ker \varphi_2\).
    求证:存在常数\(c \in F\),使得\(\varphi_1=c \varphi_2\).    
\end{problem}

\begin{proof}
    若\(\ker \varphi_1=\ker \varphi_2=V\),则\(\varphi_1=\varphi_2=0\),故\(c\)可以为任意常数.

    若\(\exists u \notin \ker \varphi,u \in V\),
    则根据\probref{3.B.29},\(\forall v \in V\),\(v=v_0+a_vu,v_0 \in \ker \varphi,a_v \in F\).
    
    从而\(\varphi_1(v)=\varphi_1(v_0+a_vu)=a_v \varphi_1(u)\)
    并且\(\varphi_2(v)=\varphi_2(v_0+a_vu)=a_v \varphi_2(u)\).
    \begin{align*}
        \dfrac{\varphi_1(v)}{\varphi_2(v)}=\dfrac{\varphi_1(u)}{\varphi_2(u)}=\mathrm{cons.}
    \end{align*}
    由于\(u \notin \ker \varphi_2\),故\(\varphi_2(u) \ne 0\),该式恒成立,证毕.    
\end{proof}
% End: source/chapter_3/3.B.tex

