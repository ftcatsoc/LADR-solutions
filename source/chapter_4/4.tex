\section{4 Polynomials}

\begin{theorem}[4.6*]\label{thm 4.6*} \footnote{\kaishu 标*的为第四版上的习题或者按第四版的顺序组织的定理.} 
    多项式的因式分解 \:
    设\(\altdeg p=m\)且\(\lambda \in F\).

    那么\(p(\lambda)=0\)当且仅当存在\(\altdeg q=m-1\),使得\(p(z)=(z-\lambda)q(z)\).
\end{theorem}

\begin{proof}
    充分性显然,下证必要性.设\(p(z)=\sum_{i=0}^m a_iz^i,a_0,\cdots,a_m \in F\).

    因此\(p(z)=p(z)-p(\lambda)=\sum_{i=0}^m a_i(z^i-\lambda^i)\).
    利用\(a^n-b^n=(a-b)\sum_{i=1}^n a^{n-i}b^{i-1}\),有
    \begin{align*}
        p(z)-p(\lambda)=\sum_{i=0}^m a_i(z^i-\lambda^i)
                        =(z-\lambda)\sum_{i=0}^m \sum_{j=1}^i a_iz^{i-j}\lambda^{j-1}
                        =(z-\lambda)\sum_{i=0}^m \sum_{j=0}^{i-1} a_{j+1}z^j\lambda^i
    \end{align*}
    令\(q(z)=\sum_{i=0}^{m-1} \sum_{j=0}^{i-1} a_{j+1}z^j\lambda^{i-j}\),则\(p(z)=(z-\lambda)q(z)\),得证.
\end{proof}

\begin{theorem}[4.8*]\label{thm 4.8*} 多项式的零点上限 \:
    设\(p \in P(F),\altdeg p=m\),那么\(p\)至多有\(m\)个零点.
\end{theorem}

\begin{proof}
    使用数学归纳法.当\(m=1\)时,显然\(p(z)\)只有一个零点\(-\dfrac{a_0}{a_1}\).

    设结论对于任意\(\altdeg q=m-1\)的\(q(z)\)均成立.当\(\altdeg p=m\)时,若\(p\)无零点,则结论成立.
    
    若\(p\)有一个零点\(\lambda\),则根据定理4.6*,存在\(\altdeg q=m-1\),使得\(p(z)=(z-\lambda)q(z)\).
    
    由于\(\altdeg q=m-1\),根据归纳假设,\(q\)至多有\(m-1\)个零点,故\(p\)至多有\(m\)个零点,得证.
\end{proof}

\begin{theorem}[4.8*]\label{col 4.8*} 推论:多项式系数的确定性 \:
    若\(p(z)=\sum_{i=0}^m a_iz^i,q(z)=\sum_{i=0}^m b_iz^i\),且\(p=q\).

    则\(\forall i=0,\cdots,m,a_i=b_i\),也即\(p\)的系数唯一确定.
\end{theorem}

\begin{proof}
    使用反证法.若\(\exists i=0,\cdots,m,a_i \ne b_i\),则\(\altdeg (p-q)>0\)且有限.

    于是\(p-q\)最多有\(\altdeg (p-q)\)个零点,而\(p-q=0\)有无穷个零点,矛盾,得证.
\end{proof}

\begin{theorem}[4.9*]\label{thm 4.9*} 多项式的带余除法 \:
    设\(p,s \in P(F),s \ne 0\),

    则存在唯一的\(q,r \in P(F)\)使得\(p=sq+r\),且满足\(\altdeg r<\altdeg s\).
\end{theorem}

\begin{proof}
    令\(\altdeg p=n,\altdeg s=m\),则根据\probref{2.C.10},\(1,\cdots,z^{m-1},s,zs,z^{n-m}s\)线性无关.

    由于该向量组长度为\(n+1=\dim P_n(F)\),故它是\(P_n(F)\)的一组基.
    
    因此\(\forall p \in P_n(F),\exists a_0,\cdots,a_{m-1},b_0,\cdots,b_{n-m},\)
    \(p=\sum_{i=0}^{m-1}a_iz^i+s\sum_{i=0}^{n-m}b_iz^i\).
    
    令\(q=\sum_{i=0}^{n-m}b_iz^i,r=\sum_{i=0}^{m-1}a_iz^i\).由系数和次数的唯一性,\(q,r\)唯一,得证.
\end{proof}
% End: source/chapter_4/4.tex

