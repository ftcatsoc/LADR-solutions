\section{2.A Span and Linear Independnce}

\begin{problem}[10]\label{2.A.10}
    设\(v_1,\cdots,v_m\)在\(V\)中线性无关且\(w\in V\).

    求证:若\(v_1+w,\cdots,v_m+w\)线性相关,则\(w\in \altspan (v_1,\cdots,v_m)\).
\end{problem}

\begin{proof}
    若\(v_1+w,\cdots,v_m+w\)线性相关,则有\(\sum_{i=1}^m a_i(v_i+w)=0,\exists i=1,\cdots,m,a_i\ne 0\).
    
    从而\(\sum_{i=1}^m a_iv_i+w\sum_{i=1}^m a_i=0\).

    \begin{enumerate}
        \item 若\(\sum_{i=1}^m a_i\ne 0\),则\(w=-\dfrac{\sum_{i=1}^m a_iv_i}{\sum_{i=1}^m a_i}\in \altspan (v_1,\cdots,v_m)\).
        \item 若\(\sum_{i=1}^m a_i=0\),则\(\sum_{i=1}^m a_iv_i=0\).
    \end{enumerate}

    而\(v_1,\cdots,v_m\)在\(V\)中线性无关,故\(a_1=\cdots=a_m=0\).从而\(\forall i=1,\cdots,m, a_i=0\),矛盾.
\end{proof}

\begin{problem}[13]\label{2.A.13}
    求证:\(V\)是无限维向量空间等价于\(V\)中存在无限多线性无关的向量.
\end{problem}

\begin{proof}
    必要性的证明是显然的,以下使用数学归纳法证明充分性.
    
    先看\(n=1\)的情况并设\(v_1 \ne 0 \in V\).
    
    由于\(V\)是无限维的,故一定存在\(v_2 \in V \notin \altspan(v_1)\),从而\(v_1,v_2\)线性无关.
    
    再假设\(n=m\)时情况成立,即\(V\)中存在\(v_1,\cdots,v_m\)线性无关.
    
    考虑\(n=m+1\)时的情况.由于\(V\)是无限维的,故一定存在\(v_{m+1} \notin \altspan(v_1,\cdots,v_m)\).
    
    根据习题2.A.11,\(v_1,\cdots,v_m,v_{m+1}\)线性无关.因此命题对任意的自然数\(m\)均成立,证毕.
    %此处使用了硬编码
\end{proof}
% End: source/chapter_2/2.A.tex

