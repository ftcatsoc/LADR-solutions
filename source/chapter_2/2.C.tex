\section{2.C Dimensions}

\begin{problem}[10]\label{2.C.10}
    设\(p_0, p_1, \dots, p_m \in P(F)\),其中\(p_i\)是次数为\(i\)的多项式,最高次系数为\(c_i\).
    
    求证:\(p_0, \dots, p_m\)是\(P_m(F)\)的一组基.
\end{problem}

\begin{proof}
    设\(\sum_{i=0}^m a_ip_i=0\).考虑商空间\(P(F)/P_i(F)\)和商变换\(\pi_i(p)=p+P_i(F)\).

    对两侧施加\(\pi_{m-1}\),得到\(\sum_{i=0}^m a_ip_i+P_i(F)=a_mc_mx^m+P_i(F)=0+P_i(F)\).

    由于\(c_m \ne 0\),故\(a_m=0\).随后依次施加\(\pi_{m-2}, \dots, \pi_{0}\),得到\(a_m=\dots=a_1=0\).

    此时只剩\(a_0c_0=0\),显然\(a_m=\dots=a_0=0\),即\(p_0, \dots, p_m\)线性无关.

    由于\(p_0, \dots, p_m\)线性无关且足够长,故\(p_0, \dots, p_m\)是\(P_m(F)\)的一组基.
\end{proof}

\begin{problem}[14]\label{2.C.14}
    设\(U_1,\cdots,U_m\)都是\(V\)的有限维子空间,且\(U_1,\cdots,U_m\)相互独立.

    求证:\(\sum_{i=1}^m U_i\)是有限维向量空间,且\(\dim \sum_{i=1}^m U_i=\sum_{i=1}^m \dim U_i\).
\end{problem}

\begin{proof}
    设\(U_i\)的一组基为\(u_1^i,\cdots,u_{a_i}^i\),
    则\(\sum_{i=1}^m U_i=\operatorname{span}(u_1^1,\cdots,u_{a_1}^1,\cdots,u_{a_m}^1,\cdots,u_{a_m}^m)\).

    因而\(\dim \sum_{i=1}^m U_i \leq \sum_{i=1}^m \dim U_i\),即\(\sum_{i=1}^m U_i\)是有限维向量空间.

    下面使用数学归纳法,先验证\(U_1+U_2\)的情况.

    由于\(\dim (U_1+U_2)=\dim U_1+\dim U_2-\dim (U_1\cap U_2)\)且\(U_1\cap U_2=\{0\}\),

    故\(\dim (U_1+U_2)=\dim U_1+\dim U_2\),从而\(n=2\)的情况成立.

    随后假设\(n=m\)时结论成立,则当\(n=m+1\)时,

    有\(\dim (\sum_{i=1}^m U_i+U_{m+1})=\dim \sum_{i=1}^m U_i+\dim U_{m+1}-\dim (\sum_{i=1}^m U_i \cap U_{m+1})\).

    由于\(n=m\)时结论成立,故\(\dim \sum_{i=1}^m U_i=\sum_{i=1}^m \dim U_i\).

    又由于\(U_i\)相互独立,故\(\sum_{i=1}^m U_i \cap U_{n+1}=\{0\}\).
    故\(\dim \sum_{i=1}^m U_i=\sum_{i=1}^m \dim U_i\),证毕.
\end{proof}
% End: source/chapter_2/2.C.tex

