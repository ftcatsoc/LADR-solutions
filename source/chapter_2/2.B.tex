\section{2.B Bases}

\begin{problem}[6]\label{2.B.6}
    若\(v_1,\cdots,v_4\)是\(V\)的一组基,证明:\(v_1+v_2,v_2+v_3,v_3+v_4,v_4\)是\(V\)的一组基.
\end{problem}

\begin{proof}
    先证明它们线性无关.令\(a_1(v_1+v_2)+a_2(v_2+v_3)+a_3(v_3+v_4)+a_4v_4=0\).

    得到\(a_1v_1+(a_1+a_2)v_2+(a_2+a_3)v_3+(a_3+a_4)v_4=0\),由\(v_1,\cdots,v_4\)线性无关,
    
    得\(a_1=0,a_1+a_2=0,a_2+a_3=0,a_3+a_4=0\),即\(a_1=a_2=a_3=a_4=0\),得证.
    
    令\(v_1+v_2=u_1,v_2+v_3=u_2,v_3+v_4=u_3,v_4=u_4\),下证\(V=\altspan(u_1,u_2,u_3,u_4)\).
    
    由于\(v_1=u_1-u_2+u_3-u_4,v_2=u_2-u_3+u_4,v_3=u_3-u_4,v_4=u_4\),于是有
    
    考虑\(\forall v=\sum_{i=1}^4 a_iv_i \in V,u=\sum_{i=1}^4 b_iu_i \in \altspan(u_1,u_2,u_3,u_4),a_i,b_i \in F\).
    \begin{align*}
        v&=a_1(u_1-u_2+u_3-u_4)+a_2(u_2-u_3+u_4)+a_3(u_3-u_4)+a_4u_4 \\
            &=a_1u_1+(a_2-a_1)u_2+(a_3-a_2+a_1)u_3+(a_4-a_3+a_2-a_1)u_4 \in \altspan(u_1,u_2,u_3,u_4) \\
        u&=b_1v_1+(b_1+b_2)v_2+(b_2+b_3)v_3+(b_3+b_4)v_4 \in \altspan(v_1,v_2,v_3,v_4)=V \\
            & V \subseteq \altspan(u_1,u_2,u_3,u_4),\altspan(u_1,u_2,u_3,u_4)\subseteq V 
            \Rightarrow V=\altspan(u_1,u_2,u_3,u_4)
    \end{align*}
    于是\(u_1,u_2,u_3,u_4\)线性无关且\(V=\altspan(u_1,u_2,u_3,u_4)\),即其确为\(V\)的一组基.    
\end{proof}

\begin{problem}[8]\label{2.B.8}
    设\(U\)和\(W\)都是\(V\)的子空间,且满足\(V=U \oplus W\).

    \(u_1,\cdots,u_m,w_1,\cdots,w_n\)分别是\(U,W\)的一组基.
    求证:\(u_1,\cdots,u_m,w_1,\cdots,w_n\)是\(V\)的一组基.
\end{problem}

\begin{proof}
    先证明\(u_1,\cdots,u_m,w_1,\cdots,w_m\)线性无关.

    设\(\exists a_1,\cdots,a_m,b_1,\cdots,b_n,\sum_{i=1}^m a_iu_i+\sum_{i=1}^n b_iw_i=0\),
    即\(v=\sum_{i=1}^m a_iu_i=-\sum_{i=1}^n b_iw_i\).
    
    这表明\(U\)中的某元素与\(W\)中的某元素相等,即\(v\in U\cap W\).
    
    而\(V=U\oplus W\),即\(V\cap W=\{0\}\),得\(\sum_{i=1}^m a_iu_i=-\sum_{i=1}^n b_iw_i=0\).
    
    由\(u_1,\cdots,u_m\)和\(w_1,\cdots,w_m\)分别线性无关,有\(u_1=\cdots=u_m=w_1=\cdots=w_n=0\),得证.
    
    再证\(V=\altspan(u_1,\cdots,u_m,w_1,\cdots,w_n)\),且\(\altspan(u_1,\cdots,u_m,w_1,\cdots,w_n) \subseteq V\)是显然的.
    
    由\(V=U\oplus W\),得\(\forall v\in V,\exists u \in \altspan(u_1,\cdots,u_m),w \in \altspan(w_1,\cdots,w_n),v=u+w\).
    
    因此\(\forall v \in V,v \in \altspan (u_1,\cdots,u_m,w_1,\cdots,w_n)\),
    即\(V \subseteq \altspan(u_1,\cdots,u_m,w_1,\cdots,w_n)\).
    
    结合\(\altspan(u_1,\cdots,u_m,w_1,\cdots,w_n) \subseteq V\),
    得到\(V=\altspan (u_1,\cdots,u_m,w_1,\cdots,w_n)\),证毕.
\end{proof}
% End: source/chapter_2/2.B.tex

